\documentclass[oneside,leqno]{book}
\usepackage{amsmath}%required
\usepackage{graphicx}%required for illustrations
\usepackage[german, french]{babel}%recommended
% requires illustration files fig[j].png, j=1,2,..6
% If you run into any kind of trouble with these graphics 
%   please try the version with LaTeX pictures.

\begin{document}
\thispagestyle{empty}
\small



The Project Gutenberg etext of Sechs Vortr\"age, by Henri Poincar\'e

\bigskip

This eBook is for the use of anyone anywhere at no cost and with
almost no restrictions whatsoever.  You may copy it, give it away or
re-use it under the terms of the Project Gutenberg License included
with this eBook or online at www.gutenberg.net

\bigskip


Title: Sechs Vortr\"age �ber ausgew\"ahlte Gegenst\"ande aus der reinen 
       Mathematik und mathematischen Physik

\bigskip

Author: Henri Poincar\'e

\bigskip

Release Date: March 5, 2005  [EBook \#15267]

\bigskip

Language: German and French

\bigskip

Character set encoding: TeX

\bigskip

*** START OF PROJECT GUTENBERG'S SECHS VORTR\"AGE ***

\bigskip



Produced by Joshua Hutchinson, K.F. Creiner and the Online Distributed
Proofreading Team. This file was produced from images generously made
available by Cornell University.






\normalsize
\newpage
\selectlanguage{german}

\frontmatter
\thispagestyle{empty}
Mathematische Vorlesungen an der Universit\"at G\"ottingen: IV

\bigskip\bigskip\bigskip\bigskip\bigskip\bigskip
\begin{center}
{\large SECHS VORTR\"AGE \\
\"UBER AUSGEW\"AHLTE GEGENST\"ANDE \\}
\bigskip

{\LARGE AUS DER REINEN MATHEMATIK\\
UND DER MATHEMATISCHEN PHYSIK \\}

\bigskip\bigskip
\normalsize auf Einladung der Wolfskehl-Kommission \\
der K\"oniglichen Gesellschaft der Wissenschaften \\
gehalten zu G\"ottingen vom 22.--28. April 1909 \\
\bigskip

von\\
\bigskip
\Large HENRI POINCAR\'E \\
\bigskip
\normalsize Mitglied der Franz\"osischen Akademie \\
Professor an der Facult\'e des Sciences \\
der Universit\"at Paris

\bigskip\bigskip\bigskip
\normalsize Mit 6 in den Text gedruckten Figuren
\vfill
{\large
Leipzig und Berlin \\
Druck und Verlag von B. G. Teubner \\ \bigskip

1910
}
\end{center}

\newpage
\thispagestyle{empty}
\mainmatter
\selectlanguage{french}

\section*{Pr{\'e}face}

L'Universit\'e de G\"ottingen a bien voulu m'inviter \`a traiter
devant un savant auditoire diverses questions d'Analyse pure, de
Physique math\'ematique, d'Astrono\-mie th\'eorique et de
Philosophie math\'ematique; les conf\'erences que j'ai faites
\`a cette occasion ont \'et\'e recueillies par quelques
\'etudiants qui ont eu la bont\'e de les r\'ediger en corrigeant
les nombreuses offenses que j'avais faites \`a la grammaire
allemande. Je leur en exprime ici toute ma reconnaissance.

Il convient \'egalement que je m'excuse aupr\`es du public de la
bri\`evet\'e avec laquelle ces sujets sont trait\'es. Je ne
disposais pour exposer chacun d'eux que d'un temps tr\`es court,
et je n'ai pu la plupart du temps que donner une id\'ee
g\'en\'erale des resultats, ainsi que des principes qui m'ont
guid\'e dans les d\'e\-monstrations, sans entrer dans les
d\'etails m\^emes de ces d\'emonstrations.

\selectlanguage{german}
\newpage
\thispagestyle{empty}
\setcounter{page}{0}
\section*{Inhaltsverzeichnis}



\textbf{Erster Vortrag.}\hfill Seite

\noindent \"Uber die Fredholmschen Gleichungen \hfill \pageref{erster}
\bigskip

\noindent\textbf{Zweiter Vortrag.}

\noindent Anwendung der Theorie der Integralgleichungen auf die
Flutbewegung des \\
Meeres \hfill \pageref{zweiter} \bigskip

\noindent\textbf{Dritter Vortrag.}

\noindent Anwendung der Integralgleichungen auf Hertzsche Wellen \hfill
\pageref{dritter} \bigskip

\noindent\textbf{Vierter Vortrag.}

\noindent \"Uber die Reduktion der Abelschen Integrale und die Theorie der
Fuchsschen \\
Funktionen \hfill \pageref{vierter} \bigskip

\noindent\textbf{F\"unfter Vortrag.}

\noindent \"Uber transfinite Zahlen \hfill \pageref{fuenfter}
\bigskip

\noindent\textbf{Sechster Vortrag.}

\noindent La m\'ecanique nouvelle \hfill \pageref{sechster}
\bigskip
${}$\vfill

\newpage


\begin{center}
\bigskip\bigskip\bigskip\label{erster}
{\Large Erster Vortrag}

\bigskip\bigskip\bigskip
{\Large \"UBER DIE FREDHOLMSCHEN GLEICHUNGEN}
\end{center}

\newpage


Die Integralgleichung \label{Integral2Art}
\[
\varphi (x)=\lambda \int\limits_a^b f(x,y) \varphi(y) dy + \psi(x) 
\tag{1}
\]
wird bekanntlich aufgel\"ost durch die Integralgleichung
derselben Art
\[
\varphi (x)=\psi(x) + \lambda \int\limits_a^b \psi(y) G(x,y)dy, 
\tag{1a}
\]
wobei
\[
G(x,y)=\frac{N(x,y;\lambda \,|\, f)}{D(\lambda \,|\, f)}
\]
gesetzt ist. $N$ und $D$ sind, wie aus der
\textsc{Fredholm}schen Theorie\label{Fredholm1} bekannt ist,
zwei ganze transzendente Funktionen in bezug auf $\lambda$. Um
ihre Entwicklung explizite hinschreiben zu k\"onnen, bezeichne
man, wie \textsc{Fredholm}, mit $f\left(
\begin{smallmatrix}
x_1, & x_2, & \ldots & x_n\\
y_1, & y_2, & \ldots & y_n
\end{smallmatrix}
\right)$ diejenige $n$-reihige Determinante,
deren allgemeines Element $f(x_i,y_k)$ ist. Setzt man dann
\[
a_n=\int\limits_a^b \int\limits_a^b \ldots \int\limits_a^b f(
\begin{smallmatrix}
x_1, & x_2, & \ldots & x_n\\
x_1, & x_2, & \ldots & x_n
\end{smallmatrix}) dx_1 \ldots dx_n,
\]
so hat man
\[
D(\lambda) = \sum_0^{\infty} \frac{(-\lambda)^n}{n!} a_n.
\]
Diese Gleichung formen wir um, indem wir die durch "`Iteration"' aus
$f(x, y)$ entstehenden Kerne heranziehen. Setzen wir zun\"achst
\[
f(x_\alpha,x_\beta)f(x_\beta,x_\gamma)\cdots
f(x_\lambda,x_\mu)f(x_\mu,x_\alpha) = 
f(x_\alpha,x_\beta, \cdots x_\lambda, x_\mu),
\]
so ist klar, da{\ss} $f(
\begin{smallmatrix}
x_1, & x_2, & \ldots & x_n\\
x_1, & x_2, & \ldots & x_n
\end{smallmatrix}
)$ die Form hat
\[
\sum \pm \prod f(x_\alpha, \ldots x_\mu),
\]
wie sofort aus der Entwicklung der Determinante hervorgeht. Sei nun
\[
b_k= \int_a^b \cdots \int_a^b f(x_\alpha, \cdots x_\mu)dx_\alpha 
\cdots dx_\mu ,
\]
wobei $k$ die Anzahl der Integrationsvariabeln $x_\alpha, \ldots
x_\mu$ bedeutet, so k\"onnen wir offenbar auch setzen
\[
b_k=\int\limits_a^b f_k(x,x)dx,
\]
wenn unter
\[
f_k(x,y)=\int\limits_a^b \dotsi \int\limits_a^b f(x, x_\alpha) 
f(x_\alpha, x_\beta) \cdots f(x_\lambda, y) dx_\alpha \cdots 
dx_\lambda
\]
der ``$k$-fach iterierte Kern'' verstanden wird.
\label{itKerne1}

Wir haben den obigen Relationen zufolge jetzt
\[
a_n=\sum \pm \prod b_k.
\]

Beachten wir nun, da{\ss} gewisse unter den in einem Produkt
$\prod b_k$ enthaltenen $b_k$ einander gleich werden k\"onnen,
da{\ss} ferner gewisse der Produkte $\prod b_k$ selbst einander
gleich sein werden, n\"amlich solche, die durch eine Permutation
der $x_i$ auseinander entstehen, so ergibt eine kombinatorische
Betrachtung f\"ur $a_n$ einen Ausdruck von der Form
\[
a_n=\sum_{a\alpha+b\beta+c\gamma+\ldots=n}
\frac{n!}{a^\alpha b^\beta c^\gamma \cdots a! b! c! \cdots}
[(-1)^{\alpha+1}b_\alpha]^a 
[(-1)^{\beta +1}b_\beta ]^b 
[(-1)^{\gamma+1}b_\gamma]^c \cdots
\]
und also
\[
D(\lambda)=\sum_{a,b,c,\ldots} \frac{1}{a!b!c! \cdots}
\left(-\frac{\lambda^\alpha b_\alpha}{\alpha}\right)^a
\left(-\frac{\lambda^\beta  b_\beta }{\beta }\right)^b
\left(-\frac{\lambda^\gamma b_\gamma}{\gamma}\right)^c
\cdots
\]
d.~h.\
\[
D(\lambda)=\prod_1^\infty e^{-\frac{\lambda^\alpha b_\alpha}{\alpha}},
\tag{2}
\]
also
\begin{align*}
\log D(\lambda) &= -\sum \frac{\lambda^\alpha b_\alpha}{\alpha},
\tag{2a}
\\
\frac{D'(\lambda)}{D(\lambda)} &= -\sum \lambda^{\alpha-1} b_\alpha.
\tag{2b}
\end{align*}
Den Z\"ahler $N(x,y;\lambda)$ der Funktion $G(x, y;\lambda)$
kann man auf analoge Weise durch die Gleichung
\[
N(x,y;\lambda)=D(\lambda) \cdot \sum \lambda^h f_{h+1}(x,y)
\tag{3}
\]
definieren. Diese Gleichungen, welche sich \"ubrigens schon bei
\textsc{Fredholm} finden, sind n\"utzlich als Ausgangspunkt
f\"ur viele Betrachtungen, wie sich nun an einigen Beispielen
zeigen wird.

Die \textsc{Fredholm}sche Methode ist unmittelbar g\"ultig nur
f\"ur solche Kerne $f(x, y)$, die endlich bleiben. Wird der Kern
an gewissen Stellen unendlich, \label{unendlKern1} so kann
dennoch der Fall eintreten, da{\ss} ein iterierter
Kern,\label{itKerne2} etwa $f_n(x, y)$, endlich bleibt.
Dann l\"a{\ss}t sich die Integralgleichung mit dem iterierten
Kerne nach \textsc{Fredholm} behandeln, und \textsc{Fredholm}
zeigt, da{\ss} die urspr\"ungliche Integralgleichung (1) sich
auf diese zur\"uckf\"uhren l\"a{\ss}t. Die Aufl\"osung wird
wieder durch eine Formel der Gestalt (1a) gegeben, nur ist jetzt
\[
G=\frac{N_1(x,y;\lambda)}{D_n(\lambda)}
\]
zu setzen, wobei
\[
D_n(\lambda)=D(\lambda^n\,|\, f_n)
\]
und
\[
N_1(x, y;\lambda) = D_n(\lambda)\cdot\sum \lambda^h f_{h+1}(x, y)
\]
ist. Dabei sind $N_1$ und $D_n$ wieder ganze transzendente
Funktionen von $\lambda$; jedoch zeigt es sich, da{\ss} sie
einen gemeinsamen Teiler besitzen;  wir wollen zusehen, wie sich
dies aus unseren Formeln (2) bis (3) ergibt und wie wir eine
Bruchdarstellung der meromorphen Funktion $G$ erhalten, bei der
Nenner und Z\"ahler ganze Funktionen ohne gemeinsamen Teiler
sind.

Aus unserer Annahme \"uber die iterierten Kerne folgt, da{\ss}
die Koeffizienten $b_n$, $b_{n+1},\ldots$ endlich sind. Bilden
wir nun in Anlehnung an Gleichung (2a) die Reihe
\[
K(\lambda)=-\lambda^n\frac{b_n}{n}-\lambda^{n+1}\frac{b_{n+1}}{n+1}- 
\cdots,
\]
so wird dieselbe konvergieren. Jetzt setzen wir
\[
G(x,y;\lambda)=\frac{e^K \sum \lambda^hf_{h+1}}{e^K}
\]
und behaupten, in dieser Formel die gew\"unschte Darstellung zu haben.

Um dies zu beweisen, haben wir zu zeigen, da{\ss} $e^K$ und $e^K
\cdot \sum \lambda^{h+1}f_{h+1}$ ganze Funktionen sind.

Zu diesem Zwecke bilden wir $\frac{dK}{d\lambda}$. Man berechnet
leicht
\[
-\frac{dK(\lambda)}{d\lambda}=
\lambda^{n-1}\int\limits_a^b\frac{N_1(x,x)}{D_n(\lambda)}dx+
\sum\limits_{k=1}^{k=n-1}\lambda^{n+k-1}
\iint\limits_a^b\frac{N_1(x,y)}{D_n}f_k(x,y)\,dx\,dy.
\]

Hieraus schlie{\ss}t man zun\"achst, da{\ss} $\frac{dK}{d\lambda}$ 
eine meromorphe Funktion von $\lambda$ ist; denn sie besitzt
h\"ochstens Pole in den Nullstellen von $D_n(\lambda)$, d.~h.\
in den Stellen $\lambda=\alpha \cdot \lambda_i$ wo $\alpha$ eine
$n$-te Einheitswurzel und $\lambda_i$ ein Eigenwert des Kernes
$f_n$ ist. Man kann nun zeigen, da{\ss} in diesen m\"oglichen
Unendlichkeitsstellen das \textsc{Cauchy}sche Residuum von
$\frac{dK}{d\lambda}$ gleich 1 oder 0 ist, je nachdem $\alpha=1$
oder $\alpha \neq 1$ genommen wird. Die hierzu geh\"orige
Rechnung wollen wir jetzt nicht durchf\"uhren;  man benutzt
dabei den Umstand, da{\ss} das f\"ur $\lambda = \lambda_k$
genommene Residuum von $\frac{N_1(x,y)}{D_n}$ gleich
$\varphi_k(x)\psi_k(y)$ ist, wo $\varphi_k$, $\psi_k$, die zu
$\lambda=\lambda_k$ geh\"origen Eigenfunktionen, den Gleichungen
\begin{align*} 
\int\limits_a^b \varphi_k(x) f_p(y,x)dx &=
\lambda_k^{-p}\varphi_k(y)\\ 
\int\limits_a^b \psi_k(z) f_p(z,y)dz &= \lambda_k^{-p}\psi_k(y) 
\end{align*}
gen\"ugen. Hieraus folgt, da{\ss} $e^{K(\lambda)}$ eine ganze 
transzendente Funktion ist, die nur an den Stellen $\lambda=\lambda_i$
verschwindet.

Betrachtet man ebenso den Z\"ahler von $G$, so sieht man
zun\"achst, da{\ss} er eine meromorphe Funktion von $\lambda$
wird, die h\"ochstens an den Stellen $\lambda = \alpha\lambda_i$
unendlich werden kann. Die Betrachtung der Residuen zeigt
jedoch, da{\ss} dies nicht geschieht, und somit, da{\ss} der
Z\"ahler $e^K \sum\lambda^h f_{k+1}$ ebenfalls eine ganze
transzendente Funktion ist. Damit ist die Reduktion des
\textsc{Fredholm}schen Bruches geleistet.

Die Reihenentwicklung f\"ur Z\"ahler und Nenner des
\textsc{Fredholm}schen Bruches in dieser reduzierten Gestalt
erhalten wir, indem wir auf die Bildungsweise von $K(\lambda)$
zur\"uckgehen; setzen wir den Nenner
\[
e^{K(\lambda)}=\sum(-\lambda)^n \frac{a'_n}{n!},
\]
so haben wir
\[
a'_n=\sum_{a\alpha+b\beta+c\gamma+\cdots=n}\pm b_\alpha^a \, 
b_\beta^b \,
b_\gamma^c\,\cdots,
\]
wobei zu setzen ist 
\[
\begin{array}[t]{ll}
  b_\alpha = 0 & \text{ f\"ur } \alpha<n \text{ und }\\
  b_\alpha=\int\limits_a^b f_\alpha(x,x)dx \quad & 
    \text{ f\"ur } \alpha \geq n.
\end{array}
\]
In analoger Weise wird der Z\"ahler gebildet. Man mu{\ss} also
die Determinanten in der gew\"ohnlichen Weise entwickeln, aber
diejenigen Glieder dieser Entwicklung wegwerfen, welche einen
Faktor von der Form $f(x_1,x_2,\ldots x_k)$ mit weniger als $n$
Ver\"anderlichen enthalten.

Unsere Formeln (2), (2a), (3) sind auch in dem Falle von Nutzen,
da{\ss} au{\ss}er dem Kern $f(x, y)$ auch alle iterierten Kerne
unendlich werden und die \textsc{Fredholm}sche Methode also nun
sicher versagt.\\ Seien etwa die Zahlen $b_1, b_2, \ldots
b_{n-1}$ unendlich, $b_n, b_{n+1}, \ldots$ endlich. Man kann
dann jedenfalls die Reihe $K(\lambda)$ bilden, fragen, ob sie
konvergiert, und untersuchen, ob $e^{K(\lambda)}$ wieder eine
ganze Funktion darstellt. Unter der Voraussetzung, da{\ss} $f(x,y)$ 
ein symmetrischer Kern ist, d.~h.
\[
f(x, y) = f(x, y), 
\]
ist mir dieser Nachweis gelungen. Ich benutze dabei die Relationen
\[
b_n = \sum \lambda_i^{-n}, 
\]
die f\"ur $n > 2$ gelten m\"ussen, da das Geschlecht der
Funktion $D(\lambda)$ einem \textsc{Hadamard}schen Satze
zufolge kleiner als 2 ist.\label{Hadamard}

Den Beweis mitzuteilen fehlt jetzt die Zeit.

F\"ur den Z\"ahler des \textsc{Fredholm}schen Bruches habe ich
die Betrachtung nicht durchgef\"uhrt.

Noch einige Worte \"uber die \label{Integral1Art}
Integralgleichung 1.~Art! Auf gewisse derartige Integralgleichungen kann man, wenn man sie zuvor auf Integralgleichungen der 2.~Art 
zur\"uckf\"uhrt, die \textsc{Fredholm}sche Methode direkt anwenden. 
Es liege z.~B.\ die Gleichung
\[
\int\limits_{-\infty}^{+\infty} \varphi(y)[e^{ixy} 
 + \lambda f(x,y)] dy = \psi(x) \qquad (-\infty <  x < +\infty)
\tag{1}
\]
vor, in der $\psi(x)$ die gegebene, $\varphi(x)$ aber die
gesuchte Funktion ist, w\"ahrend der Bestandteil $f(x, y)$ des
Kerns eine gegebene Funktion ist, die gewissen, weiter unten
angegebenen beschr\"ankenden Voraussetzungen unterworfen ist.
F\"ur die gesuchte Funktion $\varphi(y)$ machen wir den Ansatz
\[ 
\varphi(y) = \int\limits_{-\infty}^{+\infty} \Phi(z) e^{-izy} dz,  
\]
aus dem nach dem \textsc{Fourier}schen Integraltheorem,
\label{FourierIntegral} falls $\Phi(x)$ die Bedingungen f\"ur
dessen G\"ultigkeit erf\"ullt, umgekehrt
\[
2\pi \Phi(x) = \int\limits_{-\infty}^{+\infty} \varphi(y) e^{ixy} dy 
\]
folgt. Danach verwandelt sich (1) in
\[
 2\pi \Phi(x) + \lambda \int\limits_{-\infty}^{+\infty} 
\int\limits_{-\infty}^{+\infty} \Phi(z) f(x,y) e^{-izy} dz\, 
dy = \psi(x) 
\]
oder
\[
2\pi \Phi(x) + \lambda \int\limits_{-\infty}^{+\infty} 
\Phi(z) K(x,z) dz = \psi(x), 
\]
wenn
\[
K(x, z) = \int\limits_{-\infty}^{+\infty} f(x, y) e^{-izy} dy \tag{2}
\]
gesetzt wird, und damit sind wir bereits bei einer
Integralgleichung 2.~Art angelangt. Der Kern~(2) gestattet die
Anwendung der \textsc{Fredholm}schen Methode z.~B.\ dann, wenn
$f(x, y)$ und $\frac{\partial f(x,y)}{\partial y}$
gleichm\"a{\ss}ig in $x$ f\"ur $y = \pm\infty$ gegen $0$
konvergieren und die Ungleichung
\[
\frac{\partial^2 f}{\partial y^2} < \frac{M}{1 + y^2}
\]
statthat, in der $M$ eine von $x$ und $y$ unabh\"angige
Konstante bedeutet. Von $\psi(x)$ gen\"ugt es etwa, anzunehmen,
da{\ss} es nur endlichviele Maxima und Minima besitzt und im
Intervall $-\infty \cdots +\infty$ absolut integrierbar ist.

Wir k\"onnen dieselbe Methode auf eine Reihe
\[
\psi(x) = \sum_{(m)} A_m\left[ e^{imx} + \lambda\theta_m(x)
\right]
\]
anwenden; das Problem ist hier also, wenn $\psi(x)$ und die
Funktionen $\theta_m(x)$ gegeben sind, die Koeffizienten $A_m$
so zu berechnen, da{\ss} die hingeschriebene Entwicklung
g\"ultig ist. Handelte es sich soeben um eine Erweiterung des
\textit{Fourierschen Integraltheorems}, so haben wir es jetzt
mit einer Verallgemeinerung der \textit{Fourierschen Reihe} 
\label{FourierReihe} zu tun.

Setzen wir $\varphi(z)$ in der Form
\[
\varphi(z) = \sum_{(m)} A_m e^{imz};
\quad 2\pi A_m = \int\limits_0^{2\pi} \varphi(z) e^{-imz} dz
\]
an, so bekommen wir
\[
\psi(x) = \varphi(x) + \frac{\lambda}{2\pi} \int\limits_0^{2\pi} 
 \varphi(z) \cdot \sum_{(m)} e^{-imz} \theta_m(x) \cdot dz.
\]
Von der Reihe, welche hier als Kern fungiert, m\"ussen wir
voraussetzen, da{\ss} sie absolut und gleichm\"a{\ss}ig
konvergiert, d.~h.\ wir m\"ussen annehmen, da{\ss}
\[
\sum_{(m)} \left\lvert \theta_m(x) \right\rvert \tag{3}
\]
gleichm\"a{\ss}ig konvergiert.

Setzen wir beispielsweise
\[
\lambda = 1,\quad \theta_m(x) = e^{i\mu_m x} - e^{imx},
\]
so erhalten wir eine Entwicklung der Form
\[
\psi(x) = \sum_{(m)} A_m e^{i\mu_m x}.
\]
Die Bedingung (3) ist erf\"ullt, wenn wir die absolute Konvergenz von
\[
\sum_{(m)} (\mu_m - m) 
\]
voraussetzen.

Endlich betrachten wir noch die Gleichung
\[
\int\limits_0^{2\pi} \varphi(y) [e^{ixy} + \lambda f(x, y)]dy = \psi(x), \quad (-\infty < x < +\infty)
\tag{4}
\]
welche sich von (1) dadurch unterscheidet, da{\ss} das Integral
nicht in unendlichen, sondern in endlichen Grenzen zu nehmen
ist. In diesem Fall darf $\psi(x)$ nicht willk\"urlich gew\"ahlt
werden: es mu{\ss}, falls $f(x, y)$ holomorph ist, sicher eine
ganze transzendente Funktion sein, wenn die Gleichung~(4) eine
Aufl\"osung besitzen soll. Dagegen d\"urfen die Werte~$\psi(m)$
dieser Funktion~$\psi$ f\"ur alle ganzen Zahlen $m$ im
wesentlichen willk\"urlich angenommen werden. Setze ich
n\"amlich
\[
\varphi(z) = \sum_{(m)} A_m e^{-imz}, \quad\text{wo}\quad 
2\pi A_m = \int_0^{2\pi} \varphi(y) e^{imy} dy \quad\text{ ist,}
\]
so verwandelt sich (4), f\"ur $x = m$ genommen, in
\[
2\pi A_m + \lambda  \sum_{(p)} A_p \int\limits_0^{2\pi} 
 e^{-ipy} f(m, y) dy = \psi(m). 
\]
Wir gelangen so zu einem System unendlich vieler linearer
Gleichungen mit unendlich vielen Unbekannten, \label{linGl} wie
sie von \textsc{Hill}, H. v. \textsc{Koch}, \textsc{Hilbert}
\label{Hill}\label{Koch}\label{Hilbert1}u.~a.\ untersucht
worden sind. Die L\"osung dieses Systems ist, falls wir f\"ur
die Reihe
\[
 \sum_{(p, m)} \int\limits_0^{2\pi} e^{-ipy} f(m, y) dy
\tag{5}
\]
die Voraussetzung absoluter und gleichm\"a{\ss}iger Konvergenz
machen, der \textsc{Fredholm}schen L\"osung der Integralgleichungen
durchaus analog und stellt sich wie diese als meromorphe
Funktion des Parameters~$\lambda$ dar. Die gleichm\"a{\ss}ige
und absolute Konvergenz von (5) ist aber, wie sich durch
partielle Integration ergibt, sichergestellt, falls die Summe
\[
\sum_{(m)} f''(m, z)
\]
oder das Integral
\[
\int\limits_{-\infty}^{+\infty} f''(x, z) dx
\]
absolut und gleichm\"a{\ss}ig konvergiert.

Man sieht die \"Ahnlichkeit und den Unterschied der beiden
F\"alle (1) und (4) deutlich: je nachdem die Integrationsgrenzen
unendlich oder endlich sind --- oder auch, je nachdem der Kern
in den Integrationsgrenzen keine oder eine gen\"ugend hohe
Singularit\"at aufweist ---, darf man die ``gegebene'' Funktion
im wesentlichen willk\"urlich w\"ahlen oder ihr nur eine zwar
unendliche, jedoch \textit{diskrete} Reihe von Funktionswerten
vorschreiben. Es w\"are wohl nicht ohne Interesse, den hier zur
Geltung kommenden Unterschied mit Hilfe der Iteration der Kerne
n\"aher zu betrachten.

\newpage


\begin{center}
\bigskip\bigskip\bigskip\label{zweiter}\setcounter{equation}{0}
{\Large Zweiter Vortrag}

\bigskip\bigskip\bigskip\bigskip
{\Large ANWENDUNG
\bigskip

DER THEORIE DER INTEGRALGLEICHUNGEN 
\bigskip

AUF DIE FLUTBEWEGUNG DES MEERES}
\end{center}

\newpage


Ich will heute \"uber einige Anwendungen der
Integralgleichungstheorie auf die Flutbewegung berichten, die
ich im letzten Semester gelegentlich einer Vorlesung \"uber
diese Erscheinung gemacht habe. \label{Flutproblem}

Die Differentialgleichungen des Problems sind die folgenden:
\begin{equation}
\left\{
\begin{array}{l l}
\mbox{a) }&
k^2 \sum\frac{\partial}{\partial x} \left( h_1 
\frac{\partial \varphi}{\partial x}
\right) + k^2 \left( 
\frac{\partial \varphi}{\partial x}\frac{\partial h_2}{\partial y} 
- \frac{\partial \varphi}{\partial y}
\frac{\partial h_2}{\partial x}\right) = \zeta, \\
\mbox{b) }&
g \cdot \zeta = - \lambda^2 \varphi + \mathit{\Pi} + W.
\end{array}\right.
\end{equation}

Wir stellen uns dabei vor, da{\ss} die Kugeloberfl\"ache der
Erde etwa durch stereographische Projektion konform auf die
$(x,y)$-Ebene bezogen sei; dann bedeute $k(x,y)$ das
\"Ahnlichkeitsverh\"altnis der Abbildung zwischen Ebene und
Kugel. Die L\"osung des Flutproblems denken wir uns durch
periodische Funktionen der Zeit $t$ gegeben, und wir nehmen
speziell an, da{\ss} unsere Gleichungen (1) einem einzigen
periodischen Summanden von der Form $A \cos (\lambda t+ \alpha)$
entsprechen, soda{\ss} also $\lambda$ in unseren Gleichungen die
Schwingungsperiode bestimmt; es ist bequem, statt des Kosinus
komplexe Exponentialgr\"o{\ss}en einzuf\"uhren und also etwa
anzunehmen, da{\ss} alle unsere Funktionen die Form
\[
e^{i\lambda t} \cdot f(x,y)
\]
haben; der reelle und imagin\"are Teil dieser komplexen
L\"osungen stellt uns dann die physikalisch brauchbaren
L\"osungen dar.

$\varphi(x, y)$ ist definiert durch
\[
-\lambda^2 \varphi = V-p,
\]
wo $V$ das hydrostatische Potential,\label{hydroPot} $p$ der
Druck ist.

Ist $h$ die Tiefe des Meeres, so definieren wir
\begin{align*}
h_1 &= - \frac{ h\lambda^2}{-\lambda^2 + 4\omega^2 \cos^2\vartheta},
\\
h_2 &= - \frac{2\omega i\cos\vartheta}{\lambda} h_1, \qquad(i 
= \sqrt{-1})
\end{align*}
wo $\vartheta$ die Colatitude\label{Colatitude} des zu $(x,y)$
geh\"origen Punktes der Erde, $\omega$ die Winkelgeschwindigkeit
der Erde bedeutet. $\zeta(x, y)$ ist die Differenz zwischen der
Dicke der mittleren und der gest\"orten Wasserschicht, d.\,h.\
$\zeta > 0$ entspricht der Ebbe, $\zeta < 0$ der Flut.
$g$ ist die Beschleunigung der Schwerkraft, $W$ das Potential
der St\"orungskr\"afte, $\mathit{\Pi}$ ist das Potential,
welches von der Anziehung der Wassermassen von der Dicke $\zeta$
herr\"uhrt. Ist z.~B.
\begin{align*}
  \zeta &= \sum A_n X_n,\\
  \intertext{so wird}
  \mathit{\Pi} &= \sum \frac{4\pi A_n}{2n+1} X_n,
\end{align*}
wo die $X_n$ die Kugelfunktionen sind.

Die Einheiten sind so gew\"ahlt, da{\ss} die Dichte des Wassers
gleich $1$, der Radius der Erdkugel gleich $1$ ist.

Die Gr\"o{\ss}e $\mathit{\Pi}$ kann man meistens
vernachl\"assigen; tut man dies, so erh\"alt man sofort f\"ur
$\varphi$ eine partielle Differentialgleichung 2.\ Ordnung. Um
aus derselben $\varphi$ zu bestimmen, mu{\ss} man gewisse
Grenzbedingungen vorschreiben. Wir unterscheiden da zwei
F\"alle:

1.\ Der Rand des Meeres ist eine vertikale Mauer; dann wird
\[
  \frac{\partial\varphi}{\partial n}
  + \frac{2\omega i}{\lambda} \cos\vartheta
  \cdot \frac{\partial\varphi}{\partial s} = 0,
\] 
wobei $\frac{\partial\varphi}{\partial n}$,
$\frac{\partial\varphi}{\partial s}$ die normale bzw.\
tangentiale Ableitung von $\varphi$ ist.

2.\ Der Rand des Meeres ist nicht vertikal; dann ist dort
\[
  h = 0,\quad \text{also} \quad h_1 = h_2 = 0.
\]
Die Grenzbedingung lautet hier, da{\ss} $\varphi$ am Rande
regul\"ar und endlich bleiben soll.

Um auf diese Probleme die Methoden der Integralgleichungen
anwenden zu k\"onnen, erinnern wir uns zun\"achst der
allgemeinen \"Uberlegungen, wie sie \textsc{Hilbert}
\label{Hilbert2} und \textsc{Picard}\label{Picard} f\"ur 
Differentialgleichungen anstellen. Sei
\[
  D(u) = f(x,y)
\]
eine partielle Differentialgleichung 2.\ Ordnung f\"ur $u$, die
elliptischen Typus hat, so ist eine, gewisse Grenzbedingungen
erf\"ullende, L\"osung $u$ darstellbar in der Form
\[
  u = \int f'G \:d\sigma',
\]
wobei $G(x,y;x',y')$ die zu diesen Randbedingungen geh\"orige
\textsc{Green}sche Funktion\label{Green} des 
Differentialausdruckes $D(u)$ ist; $f'$ ist $f(x',y')$, $d\sigma'
= dx' \cdot dy'$, und das Integral ist \"uber dasjenige Gebiet
der $(x',y')$-Ebene zu erstrecken, f\"ur welches die
Randwertaufgabe gestellt ist. Um die \textsc{Green}sche
Funktion zu berechnen und so die Randwertaufgabe zu l\"osen,
setze man
\[
  D(u) = D_0(u) + D_1(u),
\]
wo
\[
  D_1(u) = a\frac{\partial u}{\partial x}
  + b\frac{\partial u}{\partial y} + cu
\]
ein linearer Differentialausdruck ist. Nehmen wir nun an, wir
kennen die \textsc{Green}\-sche Funktion $G_0$ von $D_0(u)$, so
haben wir die L\"osung von
\[
D(\varphi) = f
\]
in der Form
\[
\varphi = \int G_0 \left( f' - a'\frac{\partial\varphi'}{\partial x'} - b'\frac{\partial\varphi'}{\partial y'} 
- c'\varphi' \right) d\sigma' .
\]
Schaffen wir hieraus durch partielle Integrationen die
Ableitungen $\frac{\partial\varphi'}{\partial x'}$,
$\frac{\partial\varphi'}{\partial y'}$ heraus, so werden wir
direkt auf eine Integralgleichung zweiter Art f\"ur $\varphi$
gef\"uhrt, die wir nach der \textsc{Fredholm}schen Methode
\label{Fredholm2} behandeln k\"onnen, wenn ihr Kern nicht zu
stark singul\"ar wird.

Bei unserem Probleme der Flutbewegung tritt nun gerade dieser
Fall ein; der Kern wird so hoch unendlich, \label{unendlKern2}
da{\ss} die \textsc{Fredholm}schen Methoden versagen; ich will
Ihnen jedoch zeigen, in welcher Weise man diese Schwierigkeiten
\"uberwinden kann.

Betrachten wir erst den Fall der ersten Grenzbedingung
\[
\frac{\partial\varphi}{\partial n} + 
C\frac{\partial\varphi}{\partial s} = 0,
\]
wo $C$ eine gegebene Funktion von $x, y$ ist. Die
Differentialgleichung, die sich bei Vernachl\"assigung von
$\mathit{\Pi}$ ergibt, hat die Form
\[
A\Delta\varphi + D_1 = f,
\]
und wir stehen daher vor der Aufgabe, die Gleichung
\[
\Delta\varphi = F
\]
mit unserer Randbedingung zu integrieren.

Diese Aufgabe ist \"aquivalent mit der, eine im Innern der
Randkurve regul\"are Potentialfunktion $V$, die am Rande die
Bedingung $\frac{\partial V}{\partial n} + C\frac{\partial
V}{\partial s} = 0$ erf\"ullt, als Potential einer einfachen
Randbelegung zu finden. Bezeichnet $s$ die Bogenl\"ange auf der
Randkurve von einem festen Anfangspunkte bis zu einem
Punkte~$P$, $s'$~die bis zum Punkte $P'$, so erh\"alt man f\"ur
$V$ eine Integralgleichung; jedoch wird der Kern $K(s,s')$
derselben f\"ur $s = s'$ von der ersten Ordnung unendlich, und
es ist daher in dem Integrale
\[
\int_A^B K(x,y) f(y) dy
\]
der sogenannte \textsc{Cauchy}sche Hauptwert zu nehmen, der
definiert ist als das arithmetische Mittel aus den beiden
Werten, die das Integral erh\"alt, wenn ich es in der komplexen
$y$-Ebene unter Umgehung des Punktes $y = x$ das eine mal auf
einem Wege $AMB$ oberhalb, das andere mal auf einem Wege $AM'B$
unterhalb der reellen Achse f\"uhre.

Anstatt die Methoden zu benutzen, die \textsc{Kellogg}
\label{Kellog} zur Behandlung solcher unstetiger Kerne angibt,
will ich einen andern Weg einschlagen. Wir betrachten neben der
Operation
\[
S\big(f(x)\big) = \int K(x,y)f(y)dy
\]
die iterierte
\[
S^2\big(f(x)\big) = \iint K(x,z)K(z,y)f(y)dz\,dy,
\]
bei der ebenfalls das Doppelintegral als \textsc{Cauchy}scher
Hauptwert zu nehmen ist; dies soll folgenderma{\ss}en verstanden
werden: wir betrachten f\"ur die Variable $z$ die Wege $AMB$,
$AM'B$, f\"ur $y$ die Wege $APB$, $AP'B$, die zueinander liegen
m\"ogen, wie in der Figur angedeutet ist. Dann bilden wir die 4
Integrale, die sich ergeben, wenn ich einen Weg f\"ur $z$ mit
einem f\"ur $y$ kombiniere;

\begin{figure}[htb]
\begin{center}
\includegraphics*[scale=0.3,keepaspectratio]{fig1.png}
\end{center}
\end{figure}

\begin{tabular}{lrrrr}
$z:$ &$AMB$, &$AM'B$, &$AMB$, &$AM'B$ \\
$y:$ &$APB$, &$APB$,  &$AP'B$, &$AP'B$, \\
\end{tabular}

\noindent und nehmen aus diesen 4 Integralen das arithmetische
Mittel. Ziehen wir noch 2 Wege $AQB$, $AQ'B$ wie in der Figur,
so sehen wir, da{\ss} sich in der ersten Wegkombination der Weg
$AMB$ f\"ur $z$ ersetzen l\"a{\ss}t durch $AQB + AMBQA$, in der
zweiten $AM'B$ durch $AQ'B$, in der dritten $AMB$ durch $AQB$
und in der vierten $AM'B$ durch $AQ'B + AM'BQ'A$, soda{\ss} wir
jetzt die folgenden Wegkombinationen haben:

\begin{tabular}{lll}
{\centering $z$} & \quad & {\centering $y$}  \\
$AQB + AMBQA$      &       & $APB$ \\
$AQ'B$             &       & $APB$ \\
$AQB$              &       & $AP'B$ \\
$AQ'B + AM'BQ'A$   &       & $AP'B$.
\end{tabular}

F\"uhren wir jetzt die Integrale aus und wenden den
Residuenkalk\"ul auf die geschlossenen Wege an, so zeigt sich,
da{\ss} unsere Operation $S^2\big(f(x)\big)$, die einer
Integralgleichung 1.~Art zugeh\"ort, \"ubergeht in eine
Operation, welche durch die linke Seite einer Integralgleichung
2.~Art gegeben ist, deren Kern \"uberall endlich bleibt; wenn
wir zuerst die vier Kombinationen von den Wegen $AQB$ und $AQ'B$
mit den Wegen $APB$ und $AP'B$ nehmen, so bekommen wir ein
doppeltes Integral, welches nicht unendlich werden kann, da auf
diesen Wegen $x \ne y$ und $y \ne z$. Betrachten wir jetzt die
beiden Wegkombinationen $AMBQA$, $APB$ und $AM'BQ'A$, $AP'B$,
oder $AMBQA$, $APB$ und $AQ'BM'A, BP'A$, so ist leicht zu sehen,
da{\ss} $z$ eine geschlossene Kurve $AMBQA$ oder $AQ'BM'A$ um
$y$ beschreibt, und da{\ss} gleichzeitig $y$ eine geschlossene
Kurve $APBP'A$ um $x$ beschreibt. Wir d\"urfen also die
Residuenmethode anwenden, und wir bekommen ein Glied, wo die
unbekannte Funktion ohne Integralzeichen auftritt, wie in der
linken Seite einer Integralgleichung zweiter Art. Indem wir so
auf eine durchaus regul\"are Integralgleichung 2.\ Art gef\"uhrt
werden, die der \textsc{Fredholm}schen Methode zug\"anglich ist,
haben wir die Schwierigkeit bei unserem Problem \"uberwunden.

Nur ein Punkt bedarf noch der Erl\"auterung: wenn $x$ und $y$
gleichzeitig in einen der Endpunkte $A, B$ des Intervalles
hineinfallen, so versagen zun\"achst die obigen Betrachtungen,
und es scheint, als w\"aren wir f\"ur diese Stellen der
Endlichkeit unseres durch Iteration gewonnenen Kernes nicht
sicher. Dieses Bedenken wird jedoch bei unserm Problem dadurch
beseitigt, da{\ss} der Rand des Meeres, der das
Integrationsintervall darstellt, geschlossen ist, woraus sich
ergibt, da{\ss} die Punkte $A, B$ keine Ausnahmestellung
einnehmen k\"onnen.

Durch diese \"Uberlegungen ist also der Fall der vertikalen
Meeresufer erledigt.

Wir betrachten den zweiten und schwierigeren Fall, da{\ss} das
Ufer des Meeres keine vertikale Mauer ist. Dann ist am Rande
\[
h = h_1 = h_2 = 0.
\]
Da die Glieder 2.\ Ordnung unserer Differentialgleichung f\"ur
$\varphi$ durch den Ausdruck
\[
h_1\Delta\varphi
\]
gegeben sind, so ist die Randkurve jetzt eine singul\"are Linie
f\"ur die Differentialgleichung. Au{\ss}erdem werden $h_1, h_2$
gem\"a{\ss} ihrer Definition f\"ur die durch die Gleichung
\[
4\omega^2\cos^2\vartheta = \lambda^2
\]
gegebene \textit{kritische geographische Breite}
\label{kritischegeographische} $\vartheta$ unendlich. Um trotz
dieser Singularit\"aten, welche das Unendlichwerden des Kerns
$K$ zur Folge haben, das Problem durchzuf\"uhren, bin ich
gezwungen gewesen, das reelle Integrationsgebiet durch ein
komplexes zu ersetzen, indem ich $y$ in eine komplexe
Ver\"anderliche $y + iz$ verwandle; $x$ hingegen bleibt reell.

\begin{figure}[htb]
\begin{center}
\includegraphics*[scale=0.4]{fig2.png}
\end{center}
\end{figure}

Wir deuten $xyz$ als gew\"ohnliche rechtwinklige Koordinaten in
einem dreidimensionalen Raum und zeichnen den Durchschnitt $AB$
einer Ebene $x = \text{konst.}$ mit dem in der $(x,y)$-Ebene
gelegenen Meeresbecken. Entspricht $C$ der kritischen
geographischen Breite, so ist es nicht schwer, diese
Singularit\"at durch Ausweichen in das komplexe Gebiet zu
umgehen. W\"ahlen wir ferner irgend zwei Punkte $D, E$ zwischen
$A$ und $B$ und umgeben $A$, von $D$ ausgehend und dorthin
zur\"uckkehrend, mit einer kleinen Kurve und verfahren
entsprechend bei $B$ --- r\"aumlich gesprochen: umgeben wir die
Randkurve mit einem ringf\"ormigen Futteral ---, so stellen wir
uns jetzt das Problem, unsere Differentialgleichung so zu
integrieren, da{\ss} $\varphi$, wenn wir seine Wert\"anderung
l\"angs der den Punkt $A$ umgebenden Kurve verfolgen, mit
demselben Wert nach $D$ zur\"uckkehrt, mit dem es von dort
ausging. Diese ``ver\"anderte'' Grenzbedingung ist mit der
urspr\"unglichen, welche verlangte, da{\ss} $\varphi$ am Rande
(im Punkte $A$) endlich bleibt und sich regul\"ar verh\"alt,
\"aquivalent. Zwar sind die zu der neuen und der alten
Grenzbedingung geh\"origen \textsc{Green}schen Funktionen $G$,
$G_1$ nicht identisch, wohl aber die den betreffenden
Randbedingungen unterworfenen L\"osungen von
\[
D(u) = f. \tag{1}
\]
Hiervon \"uberzeugen wir uns leichter im Falle nur
\textit{einer} Variablen $y$; dann ergeben die Gleichungen
\[
u = \int G(y,y')f(y')dy',\quad u_1 = \int G_1(y,y')f(y')dy'
\]
durch Anwendung des \textsc{Cauchy}schen Integralsatzes, da{\ss}
$u - u_1 = 0$ ist.

Um jetzt das Problem~(1) zu behandeln, ziehe ich die vorige
Methode heran, die hier aber in zwei Stufen zur Anwendung kommt,
da unsere ver\"anderte Randbedingung f\"ur die Gleichung $\Delta
u = f$ unzul\"assig ist.\footnote{
  Diese Randbedingung ist nicht von solcher Art, da{\ss} sie eine 
  bestimmte L\"osung von $\Delta(u) = f$ auszeichnet.} 
Wir k\"onnen setzen
\[
D(u) = \Delta(h_1 u) + D_1(u) + D_2(u);
\]
dabei soll $D_1(u)$ nur die Glieder 1.~Ordnung $\frac{\partial
u}{\partial x}, \frac{\partial u}{\partial y}, D_2(u)$ aber nur
$u$ selbst enthalten. Indem wir
\[
\Delta(v) = f
\]
unter der Randbedingung $v = 0$ integrieren, erhalten wir f\"ur
$u = \frac{v}{h_1}$ eine am Rande endliche und regul\"are
Funktion, f\"ur welche
\[
\Delta(h_1 u) \equiv D_0(u) = f
\]
ist. Darauf integrieren wir
\[
D_0(u) + D_2(u) = f
\]
unter Zugrundelegung der urspr\"unglichen Grenzbedingung nach
der gew\"ohn\-lichen Methode. Der in der hierbei zu benutzenden
Integralgleichung auftretende Kern ist zwar unendlich, aber von
solcher Ordnung, da{\ss} sich die Singularit\"at durch Iteration
des Kerns beseitigen l\"a{\ss}t: die partielle Integration,
welche Glieder von einer zu hohen Ordnung des Unendlichwerdens
einf\"uhren w\"urde, bleibt uns an dieser Stelle erspart.

Das damit bew\"altigte Integrationsproblem ist aber der Integration
von
\[
D_0(u) + D_2(u) = f
\]
unter der ver\"anderten Grenzbedingung \"aquivalent, und
infolgedessen k\"onnen wir jetzt die zweite Stufe ersteigen und
auch die L\"osung von
\[
D(u) \equiv \big(D_0(u) + D_2(u)\big) + D_1(u) = f
\]
unter der ver\"anderten Grenzbedingung bestimmen.

\bigskip

Wir haben bis jetzt das Glied $\mathit{\Pi}$ als so klein
vorausgesetzt, da{\ss} wir es ganz vernachl\"assigen durften.
Heben wir diese Voraussetzung auf, so entstehen keine
wesentlichen neuen Schwierigkeiten. $\mathit{\Pi}$ ist ein von
$\zeta$ erzeugtes Anziehungspotential; wir haben also
\[
\mathit{\Pi} = \int\frac{\zeta' d\sigma'}{r},
\]
wenn $d\sigma'$ ein Fl\"achenelement der Kugel, $\zeta'$ den
Wert der Funktion $\zeta$ im Schwerpunkt $(x', y')$ dieses
Fl\"achenelementes, $r$ aber die r\"aumlich gemessene Entfernung
der beiden Kugelpunkte $(x, y)$; $(x', y')$ bedeutet, und die
Integration \"uber die ganze Kugeloberfl\"ache erstreckt wird.
Wir k\"onnen auch schreiben
\[
\mathit{\Pi} = \int\frac{\zeta' dx' dy'}{k^2 r}.
\]

Setzen wir dies in unsere Ausgangsgleichungen ein, von denen wir
noch die erste mittels Aufstellung der zugeh\"origen
\textsc{Green}schen Funktion und unter Ber\"ucksichtigung der
Randbedingung aus einer Differential- in eine Integralgleichung
verwandeln, so erhalten wir zwei simultane Integralgleichungen
f\"ur $\zeta$ und $\varphi$, die mit Hilfe der soeben
er\"orterten Methoden aufgel\"ost werden k\"onnen.

\newpage


\begin{center}
\bigskip\bigskip\bigskip\label{dritter}
{\Large Dritter Vortrag}

\bigskip\bigskip\bigskip
{\Large ANWENDUNG DER INTEGRALGLEICHUNGEN \bigskip

AUF HERTZSCHE WELLEN}
\end{center}

\newpage

Ich will heute \"uber eine Anwendung der Integralgleichungen auf
\textsc{Hertz}sche Wellen vortragen und insbesondere die
\"au{\ss}erst merkw\"urdigen Beugungserscheinungen behandeln,
welche bei der drahtlosen Telegraphie eine so wichtige Rolle
spielen; ist es doch eine wunderbare Tatsache, da{\ss} die
Kr\"ummung der Erdoberfl\"ache, welche eine Fortpflanzung des
Lichtes verhindert, f\"ur die Ausbreitung der
\textsc{Hertz}schen Wellen kein Hindernis darstellt, da{\ss}
dieselben vielmehr auf der Erdoberfl\"ache von Europa bis
Amerika zu laufen verm\"ogen. Der Umstand, da{\ss} die
\textsc{Hertz}schen Wellen eine viel gr\"o{\ss}ere L\"ange haben
als die Lichtwellen, kann allein diese Erscheinung noch nicht
erkl\"aren. Eine solche Erkl\"arung ergibt sich vielmehr erst
durch Betrachtung der Differentialgleichungen des Problems.

Setzen wir die Lichtgeschwindigkeit gleich $1$, und verstehen wir
mit \textsc{Maxwell}\label{Lichtgeschwindig}\label{Maxwell}

\begin{tabular}{rrrrl}
\text{unter} & $\alpha$, & $\beta$, & $\gamma$ & 
  \text{die Komponenten der magnetischen Kraft,\label{Magnet}} \\
\text{unter} & $F,$ & $G,$ & $H$ &
  \text{die Komponenten des Vektorpotentiales,} \\
\text{unter} & $f,$ & $g,$ & $h$ &
 \text{die Komponenten der elektrischen Verschiebung,\label{Versch}} \\
\text{unter} & & $\psi$ & & 
 \text{das skalare Potential,\label{skalaresPot}} \\
\text{unter} & $u,$ & $v,$ & $w$ &
  \text{die Komponenten des Konduktionsstromes, \label{Konduk}} \\
\text{unter} & & $\varrho$ & & 
  \text{die Dichte der Elektrizit\"at, \label{Dichte}}
\end{tabular}

\noindent so gelten die Gleichungen
\[
\alpha = \frac{\partial H}{\partial y} - \frac{\partial G}{\partial x}
\]
\vspace{5mm}
\[
4\pi f = - \frac{\partial F}{\partial t} - 
\frac{\partial\psi}{\partial x},
\]
\vspace{5mm}
\[
4\pi\left(\mu + \frac{\partial f}{\partial t}\right) = 
\frac{\partial\gamma}{\partial y} - \frac{\partial\beta}{\partial z},
\]
\vspace{5mm}
\[
\sum\frac{\partial f}{\partial x} = \frac{\partial f}{\partial x} 
+ \frac{\partial g}{\partial y} + \frac{\partial h}{\partial z} 
= \varrho,
\]
\[
\sum\frac{\partial F}{\partial x} + \frac{\partial\psi}{\partial t} 
= 0,
\]
und es folgt
\begin{align*}
4\pi\cdot\mu &= \frac{\partial^2{F}}{\partial{t^2}}-\Delta F,\\
4\pi\cdot\varrho &= \frac{\partial^2{\psi}}{\partial{t^2}}-\Delta\psi.
\end{align*}
Wir betrachten nun eine ged\"ampfte synchrone \label{synchron}
Schwingung, indem wir annehmen, da{\ss} alle unsere Funktionen
proportional sind mit der Exponentialgr\"o{\ss}e
\[
e^{i\omega t} .
\]
Aus den so zustande kommenden komplexen L\"osungen erhalten wir
die physikalischen durch Trennung in reellen und imagin\"aren
Bestandteil.  Der reelle Teil von $\omega$ gibt die
Schwingungsperiode, der imagin\"are die D\"ampfung.

Aus unserem Ansatz folgt
\begin{align*}
  \frac{\partial F}{\partial t}&=i \omega \cdot F,\\
  \frac{\partial \psi}{\partial t}&=i \omega \cdot \psi,
\end{align*}
und man kann daher $F$ und $\psi$ als retardierte Potentiale
\label{retardPot} darstellen wie folgt:
\begin{align*}
F    &= \int\mu'  \frac{e^{-i \omega r}}{r} d\tau',\\
\psi &= \int\varrho' \frac{e^{-i \omega r}}{r} d\tau';
\end{align*}
$d\tau'$ ist das Raumelement im $(x', y', z')$-Raume, $\mu'$,
$\varrho'$ die Werte von $\mu$, $\varrho$ im Punkte $(x', y',
z')$, $r$ die Entfernung der Punkte $(x', y', z')$ und $(x, y,
z)$.

In den meisten Problemen treten zwei verschiedene Medien auf,
der freie \"Ather und die leitenden K\"orper; von den letzteren
wollen wir annehmen, da{\ss} sie sich wie vollkommene Leiter
verhalten, da{\ss} also in ihrem Innern das Feld verschwindet,
die elektrischen Kraftlinien auf ihrer Oberfl\"ache normal
stehen, w\"ahrend die magnetischen in dieselbe hineinfallen; dem
Umstande, da{\ss} Ladung und Str\"omung nur an der Oberfl\"ache
des Leiters vorhanden ist, wollen wir dadurch entsprechen,
da{\ss} wir die obigen Ausdr\"ucke f\"ur $F$ und $\psi$
modifizieren, indem wir an Stelle der Raumintegrale
Oberfl\"achenintegrale einf\"uhren. Wir schreiben
\begin{align*}
  \psi &= \int \varrho'' \frac{e^{-i \omega r}}{r} d\sigma',\\
  F    &= \int \mu''  \frac{e^{-i \omega r}}{r} d\sigma',
\end{align*}
wo $\varrho''$, $\mu''$ jetzt die Fl\"achendichte der Ladung
bzw.\ Str\"omung bedeuten und $d\sigma'$ das Fl\"achenelement
ist. 

Wir unterscheiden gew\"ohnlich zwei leitende K\"orper, der eine
soll der \"au{\ss}ere, der andere der innere Leiter hei{\ss}en;
sie erzeugen das ``\"au{\ss}ere'' resp.~das ``innere'' Feld; das
\"au{\ss}ere Feld ist gegeben, das innere gesucht. So ist z.~B.,
wenn wir das Problem des Empfanges elektrischer Wellen
betrachten, der Sender der \"au{\ss}ere, der Empfangsapparat der
innere Leiter; beim Probleme der Beugung elektrischer Wellen ist
der Erreger der \"au{\ss}ere, die Erdkugel der innere Leiter;
bei dem Probleme der Schwingungserzeugung haben wir kein
\"au{\ss}eres Feld, der Erreger wird dann als innerer Leiter
anzusehen sein.

Um nun zum Ansatz einer Integralgleichung zu gelangen, wollen
wir unter den oben erkl\"arten Funktionen nur die zum
unbekannten inneren Felde geh\"origen verstehen, soda{\ss} z.B.
die obigen Integrale nur \"uber die Oberfl\"ache des inneren
Leiters zu erstrecken sind; beachten wir nun, da{\ss} die innere
Normalkomponente des elektrischen Vektors am inneren Leiter
unserer obigen Annahme zufolge verschwinden mu{\ss}, so folgt,
wenn $l$, $m$, $n$ die Richtungskosinus der Normale bedeuten,
aus unseren Ausgangs-Gleichungen:
\[
4 \pi f = \frac{\partial \psi}{\partial n} + i \omega 
\left( lF + mG + nH \right) = N,
\]
wo $N$ die Normalkomponente des \"au{\ss}eren Feldes, also eine
bekannte Funktion ist.

Bezeichnen wir jetzt die Fl\"achendichte statt mit $\varrho''$
mit $\mu'$, so wird zufolge unseres Ausdruckes f\"ur $\psi$
\[
\frac{\partial \psi}{\partial n} = 2 \pi \mu + \int \mu' 
\frac{\partial}{\partial n}
\left(\frac{ e^{-i \omega r}}{r} \right) d\sigma'.
\]
Benutzen wir ferner unseren Ausdruck f\"ur $F$ und die entsprechenden
f\"ur $G$ und $H$, so hat man
\[
i \omega \sum l F = \int \frac{e^{-i \omega r}}{r} i \omega 
\sum l \mu'' \:d \sigma'.
\]
Diesen Ausdruck kann man nun in gewissen F\"allen durch partielle
Integrationen auf die Form
\[
-i \omega \int L \mu' \:d \sigma'
\]
bringen, wobei $L$ eine bekannte Funktion ist. So haben wir
schlie{\ss}lich
\[
2 \pi \mu + \int \mu' \left\{ \frac{\partial}{\partial n} 
\left( \frac{e^{-i \omega r}}{r} \right) - i \omega L \right\} 
d \sigma' = N,
\]
und dies ist die Integralgleichung 2. Art f\"ur $\mu$, auf die
wir hinstrebten.  Im allgemeinsten Falle bekommt man zwei
Integralgleichungen mit zwei Unbekannten, welche z.~B.\ $\mu$
und $\nu$ sein m\"ogen, wo $\mu$ das oben definierte ist; wir
setzen $\nu = \frac{dN}{dn}$, wo $\frac{d}{dn}$ die Ableitung in
der Normalrichtung bezeichnet und $N$ die Normalkomponente der
magnetischen Kraft ist.

\begin{figure}[htb]
\begin{center}
\includegraphics*[scale=0.3]{fig3.png}
\end{center}
\end{figure}

Die Funktion $L$ l\"a{\ss}t sich dann besonders einfach bilden,
wenn der innere Leiter ein Rotationsk\"orper ist und das
\"au{\ss}ere Feld Rotationssymmetrie besitzt. Ist $s$, $s'$ die
Bogenl\"ange, gemessen vom Endpunkte der Rotationsachse auf
einem Meridian bis zu den Punkten $P$, $P'$, ist ferner
$\vartheta$ der Winkel zwischen der Normale in $P$ und der
Meridiantangente in $P'$, so wird $L$ als Funktion von
$\vartheta$, $s$, $s'$ definiert durch die Differentialgleichung
\[
\frac{\partial L}{\partial s'} = \frac{e^{-i \omega r}}{r} 
\cos \vartheta.
\]

Das Problem des Empfanges elektrischer Wellen l\"a{\ss}t sich auf
Grund der obigen Integralgleichung 2.\ Art behandeln.

Wollen wir nur das Problem der Erzeugung elektrischer Wellen
betrachten, so haben wir das \"au{\ss}ere Feld gleich Null zu
setzen, es wird also $N=0$, und wir haben eine homogene
Integralgleichung vor uns; in ihr darf jedoch $\omega$ nicht
mehr einen willk\"urlichen Parameterwert bedeuten, sondern ist
eine zu bestimmende Zahl, die die Rolle der Eigenwerte spielt.

Ich schreibe unsere Integralgleichung in der Form
\[
2 \pi \mu + \int K \mu' \:d \sigma' = N
\]
mit dem Kerne $K$; ich f\"uhre einen unbestimmten Parameter
$\lambda$ ein und betrachte die allgemeine Gleichung
\[
2 \pi \mu + \lambda \int K \mu' \:d \sigma' = N .
\]
Das erste Glied h\"angt von zwei Unbestimmten $\lambda$ und
$\omega$ ab. Wenn man die gew\"ohnliche \textsc{Fredholm}sche
Methode\label{Fredholm3} anwendet, so erh\"alt man die L\"osung
unserer obigen Integralgleichung in Gestalt einer meromorphen
Funktion von $\lambda$, deren Nenner eine ganze Funktion von
$\lambda$ ist. Man kann nun zeigen, da{\ss} dieser Nenner auch
eine ganze Funktion von $\omega$ wird, soda{\ss} also auch hier
unsere ausgezeichneten Werte $\omega$ sich als Nullstellen einer
ganzen transzendenten Funktion ergeben.

Wir wollen aber jetzt das gr\"o{\ss}ere Problem der Beugung
ausf\"uhrlicher behandeln.

\begin{figure}[htb]
\begin{center}
\includegraphics*[scale=0.3]{fig4.png}
\end{center}
\end{figure}

Nehmen wir zu diesem Ende an, da{\ss} der innere Leiter eine
Kugel, die Erdkugel, vom Radius $\varrho$ ist und das
\"au{\ss}ere Feld (dessen normale Komponente $N$ bedeutet) von
einem punktf\"ormigen Erreger $S$ herr\"uhrt, dessen Entfernung
$D$ vom Mittelpunkt $O$ der Erde nur sehr wenig gr\"o{\ss}er ist
als der Radius $\varrho$. Wir w\"ahlen die Richtung $OS$ zur
$z$-Achse und bezeichnen die Abweichung der Richtung $OM$, in
der $M$ einen variablen Punkt der Kugeloberfl\"ache bedeutet,
von $OS$ mit $\varphi$. Die Bedeutung von $\vartheta$, $\xi$,
$\varphi'$; $r$, $r'$ ist aus der Figur ersichtlich:
\begin{align*}
	OM = OM' = OM_1 &= \varrho,\\
	OS &= D,\\
	SM &= r,\\
	SM'&=r'.
\end{align*}

Der Wert der normalen Ableitung $N$ des \"au{\ss}eren Feldes
berechnet sich im Punkte $M$, wie leicht zu sehen, nach der
Formel
\[
4 \pi N = e^{i \omega \left( t-r \right) } 
\left[ \frac{i \omega}{r} \sin \vartheta \sin \xi +
\left( \frac{1}{r^2} + \frac{1}{i \omega r^3} \right)
\cdot
\left( \sin \vartheta \sin \xi + 2 \cos \vartheta \cos \xi \right) 
\right].
\]
Da $\omega$ eine sehr gro{\ss}e Zahl ist --- denn die L\"ange
der \textsc{Hertz}schen Wellen ist klein gegen\"uber dem Radius
der Erde --- gen\"ugt es meistens, in dieser Formel nur das
erste Glied, das in der eckigen Klammer auftritt, beizubehalten.

Im vorhergehenden haben wir die Gleichung der \textsc{Hertz}sehen
Wellen auf die Form
\[
2 \pi \mu = \int \mu' K d \sigma' + N
\]
gebracht und haben gezeigt, wie der Kern $K$ berechnet werden
kann. Entwickeln wir jetzt $N$ und $K$ nach Kugelfunktionen oder
vielmehr, da unser Problem die Symmetrie eines
Rotationsk\"orpers mit der Achse $OS$ besitzt, nach
\textsc{Legendre}schen Polynomen\label{Legendre} $P_n$, so
gewinnen wir aus dieser Integralgleichung die elektrische
Fl\"achendichte $\mu$ gleichfalls unter der Form einer nach den
Funktionen $P_n$ fortschreitenden Reihe. Es gilt zun\"achst
\[
N = \sum K_n P_n; \qquad\qquad 
\int\limits_0^{\pi} P_n N \sin \varphi \:d \varphi
= \frac{2 K_n}{2n+1}.
\]
$K_n$ ist von der Form 
\[
\frac{A_n J_n\left( \omega \varrho \right)}{\varrho^2},
\]
wo $A_n$ eine nur von $n$, nicht aber von $\varrho$ abh\"angige
Zahl ist, und $J_n$ eine mit der \textsc{Bessel}schen
\label{Bessel} verwandte Funktion bedeutet.

Wir verstehen n\"amlich unter $J_n(x)$ die in der Umgebung von
$x = 0$ holomorphe L\"osung der Gleichung
\[
\frac{d^2 y}{dx^2} + y \left( 1 - 
\frac{n \left( n+1 \right)}{x^2}
\right) = 0,
\]
und $I_n(x)$ sei dasjenige Integral derselben Gleichung, welches
sich f\"ur gro{\ss}e positive Werte von $x$ angen\"ahert wie
$e^{-ix}$ verh\"alt. Da $J_n$, $I_n$ von einander unabh\"angig
sind, k\"onnen wir au{\ss}erdem daf\"ur sorgen, da{\ss}
\[
{I_n}' {J_n} - {J_n}' {I_n} = 1
\]
ist, wenn unter ${J_n}'$, ${I_n}'$ die Ableitungen von $J_n$,
$I_n$ verstanden werden.

Die L\"osung unserer Integralgleichung lautet jetzt
\[
\mu = A \sum \frac{K_n P_n \left( \cos \varphi \right)}{{I_n}' 
\left( \omega\varrho \right) J_n \left( \omega\varrho \right)}.
\]
Da aber auch der Ausdruck von $K_n$ im Z\"ahler $J_n \left(
\omega \varrho \right)$ als Faktor enth\"alt, und sich
infolgedessen dieser Term $J_n \left( \omega \varrho \right)$
heraushebt, ist
\[
{I_n}' \left( \omega \varrho \right) = 0
\]
die f\"ur die \textit{Eigenschwingungen} charakteristische Gleichung.

Um zu \"ubersichtlichen Resultaten zu gelangen, benutzen wir
angen\"aherte Formeln. Diese beruhen darauf, da{\ss} $\omega$
sehr gro{\ss}, andererseits aber $\frac{D}{\varrho}-1$ sehr
klein ist. Wir st\"utzen uns auf die folgende N\"aherungsformel
\[
\int \eta e^{i \omega \theta} dx = \eta e^{i \theta}
\sqrt{\frac{2 \pi}{\omega\theta''}} e^{\pm\frac{i \pi}{4}},
\]
$\theta$, $\eta$ sind gegebene Funktionen von $x$, $\omega$ eine
sehr gro{\ss}e Zahl, $\theta''$ bedeutet die zweite Ableitung
von $\theta$, und auf der rechten Seite ist als Argument ein
solcher Wert einzusetzen, f\"ur den $\theta$ ein Maximum oder
Minimum besitzt; je nachdem der eine oder der andere Fall
vorliegt, ist in dem Faktor $e^{\pm \frac{i \pi}{4}}$ das
Zeichen $+$ oder das Zeichen $-$ zu nehmen. Hat $\theta$ in dem
Intervall, \"uber welches zu integrieren ist, mehrere Maxima
oder Minima, so ist der Ausdruck rechts durch eine Summe analog
gebildeter Terme zu ersetzen.

Durch Anwendung dieser Formel bekommen wir f\"ur die
\textsc{Legendre}schen Polynome $P_n \left( \cos \varphi
\right)$ die folgenden, f\"ur gro{\ss}e $n$ g\"ultigen
angen\"aherten Ausdr\"ucke:
\[
P_n = 2 \sqrt \frac{2 \pi}{ n \sin \varphi}
\cdot \cos \left( n \varphi+ \frac{\varphi}{2}-\frac{\pi}{4} \right).
\]
Aus ihnen folgt f\"ur die $K_n$, falls $n<\omega\varrho$,
\[
K_n = \frac{2n+1}{8r \sqrt{n}}
\left[ e^{i \alpha} + e^{i \alpha'} \right]
\frac{i \omega \sin \vartheta \sin \xi}
{\sqrt{D \varrho \cos \vartheta \cos \xi}}
\sqrt{\frac{\sin \vartheta}{\omega \varrho}} .
\]
Dabei ist
\begin{align*}
  \alpha &= n\varphi - \omega r + \frac{\varphi}{2} - \frac{\pi}{2},\\
  \alpha'&= n\varphi' - \omega r' + \frac{\varphi'}{2} ,
\end{align*}
gesetzt, und f\"ur $\xi$, $\vartheta$, $\varphi$, $\varphi'$,
$r$, $r'$ sind die aus der Figur zu entnehmenden Werte
einzusetzen, f\"ur welche
\begin{align*}
\sin \xi = \frac{n}{\omega \varrho} && \left(\xi < \frac{\pi}{2} 
\right)
\end{align*}
wird. Die gleiche N\"aherungsformel gilt auch f\"ur $n>\omega
\varrho$, falls in der eckigen Klammer $e^{i\alpha} +
e^{i\alpha'}$ durch $e^{i\alpha}$ oder $e^{i\alpha'}$ ersetzt
wird; die Diskussion dar\"uber, welches der beiden Glieder
beizubehalten ist, will ich hier nicht geben.

Auch um ${I_n}'J_n$ angen\"ahert zu berechnen, m\"ussen wir die
beiden F\"alle $n < \omega\varrho$ und $n > \omega\varrho$
unterscheiden. Im ersten Falle ist
\[
{I_n}'J_n = e^{i \frac{\alpha-\alpha'}{2}}
\cdot
\cos \tfrac{\alpha-\alpha'}{2},
\]
im zweiten
\[
{I_n}'J_n=\tfrac{1}{2}
\]
zu setzen. Daraus ergibt sich, da{\ss} sowohl f\"ur
$n<\omega\varrho$ als auch f\"ur $n > \omega\varrho$ und
gro{\ss}e $n$
\[
\frac{K_n}{{I_n}'J_n} =\frac{\sqrt n}{2r} \: e^{i\alpha} \:
\frac{i \sqrt{\omega} \sin \xi {\left( \sin \vartheta
\right)^{\tfrac{3}{2}}}}{\varrho 
\sqrt{D \cos \vartheta \cos \xi}}
\ \footnote[1]{Der Ausdruck von $\mu$, kann auch auf eine einfachere
Form zur\"uckgef\"uhrt werden, n\"amlich
\[
\mu=\frac{-i}{4\pi\omega^2\varrho^2 D^2} \sum{ n \left( n+1
\right) \left( 2n+1 \right) \frac{I_n \left( \omega D \right)}
{{I_n}' \left( \omega \varrho \right)} P_n \left( \cos \varphi
\right)}
\]
und diese Formel ist nicht eine angen\"aherte, sondern eine strenge.}
\]
gilt. In der Summe, durch welche wir $\mu$ dargestellt haben,
geben demnach diejenigen Glieder, f\"ur welche nahezu $n = \mu$
ist, den Ausschlag.  F\"ur diese gilt n\"aherungsweise
\[
\xi=\frac{\pi}{2}\qquad\text{und}\qquad r=\sqrt{2 \varrho D}.
\]
Da ferner wegen der Kleinheit von $\frac{D}{\varrho}-1$ der Winkel
$\varphi$ immer nahezu $= 0$ bleibt, variiert $\alpha$ als
Funktion von $n$ nur sehr wenig, wenn $n$ auf die dem Werte $n =
\omega$ benachbarten ganzen Zahlen beschr\"ankt wird. Wir
d\"urfen also, wenn wir noch die L\"angeneinheit so gew\"ahlt
denken, da{\ss} $\varrho=1$ ist, schreiben
\[
\mu = C \sum
{
\frac
{\sqrt{\omega} \sin \xi {\left( \sin \vartheta \right)}^{\frac{3}{2}}}
{\sqrt{\cos \vartheta \cos \xi}}
}
\cdot
\frac{1}{\sqrt{\sin \psi}}
\left( \cos n\psi+\frac{\psi}{2}-\frac{\pi}{4} \right).
\]
Dabei ist $\mu$ der Wert der elektrischen Oberfl\"achendichte im
Punkte $M_1$ (s.~die Figur).

Aus
\begin{align*}
	\sin\xi=\frac{n}{\omega},&& \sin\vartheta=\frac{n}{\omega D};&& \cos\xi=\sqrt{1-\frac{n^2}{\omega^2}},&& \cos\vartheta=\sqrt{1-\frac{n^2}{D^2\omega^2}}
\end{align*}
bekommen wir
\[\frac
{\sin \xi {\left( \sin\vartheta \right)}^{\frac{3}{2}}}
{\sqrt{\cos\vartheta\cos\xi}}
= \frac{
\frac{n}{\omega}
\cdot
{\left(\frac{n}{\omega D}\right)}^{\frac{3}{2}}\sqrt{D}}
{\sqrt[4]{\left(1+\frac{n}{\omega}\right)\left(
 1+\frac{n}{D\omega}\right)}}
\cdot
\frac{\sqrt{\omega}}
{\sqrt[4]{\omega-n}\cdot
\sqrt[4]{\omega \left( D - 1 \right) }}
\cdot
\frac{1}{\sqrt[4]{1+\frac{\omega-n}{\omega \left( D - 1 \right)}}},
\]
soda{\ss} in der N\"ahe von $n = \omega$ der linke Ausdruck von
derselben Gr\"o{\ss}enordnung ist wie
\[
\frac{\sqrt[4]{\omega}}{\sqrt[4]{D-1}}
\cdot
\frac{1}{\sqrt[4]{n-\omega}}.
\]
F\"uhren wir diese Ann\"aherung in unsere Formel f\"ur $\mu$ ein
und ersetzen \\$\cos{\left( n \psi+\frac{\psi}{2} -
\frac{\pi}{4}\right)}$ zun\"achst durch $e^{i\left(n\psi+ 
\frac{\psi}{2}-\frac{\pi}{4}\right)}$, so kommen wir auf die 
Reihe
\[
\frac{\omega^{\tfrac{3}{4}}e^{i\left(\frac{\psi}{2}-
\frac{\pi}{4}\right)}}{\sqrt{\sin\psi}\cdot\sqrt[4]{D-1}}
\cdot
\sum_{\left(n\right)}
{\frac{e^{i n \psi}}{\sqrt[4]{n-\omega}}}.
\]
Schreiben wir
\[
S=\sum_{\left(n\right)}{\frac{e^{i n \psi}}{\sqrt{n-\omega}}},
\]
so k\"onnen wir
\begin{align*}
&&&&\int\limits_{\nu}^{\nu+1}{S e^{-i \omega \psi} d \omega}&&
\text{($\nu$ ganzzahlig})
\end{align*}
als einen Mittelwert der Reihe $S$ betrachten, und ich will $S$
durch diesen Mittelwert ersetzen. Ein solches Verfahren ist
gewi{\ss} berechtigt, wenn es uns nur daran liegt, die
Gr\"o{\ss}enordnung von $S$ festzustellen, umsomehr als in
Wirklicheit von einer Antenne nicht blo{\ss} Schwingungen einer
einzigen Wellenl\"ange, sondern ein ganzes kontinuierliches
Spektrum von Schwingungen ausgeht. Wir erhalten
\begin{align*}
\int\limits_{\nu}^{\nu+1}{S e^{-i \omega \psi} d \omega}
&=\sum_{\left(n\right)}{\int\limits_{\nu}^{\nu+1}{
\frac{e^{i\left(n-\omega\right)\psi}}
{\sqrt[4]{n-\omega}}
d \omega
}}\\
&=-\int\limits_{-\omega}^{\infty}{
\frac{e^{i q \psi}}{\sqrt[4]{q}}
}d q,
\end{align*}
und da $\omega$ sehr gro{\ss} ist, wird dieses Integral mit
$\displaystyle\int\limits_{-\infty}^{+\infty}
\displaystyle\frac{e^{i q \psi}}{\sqrt[4]{q}}d q$ im
wesentlichen \"ubereinstimmen.

Auf \"ahnliche Weise zeigt man, da{\ss} der Mittelwert von
\[
\sum{\frac{e^{-i n \psi}}{\sqrt[4]{n-\omega}}}
\]
gegen den von $S$ zu vernachl\"assigen ist. Damit gewinnen wir das
Resultat, da{\ss}
\[
\mbox{$\mu$ von der Gr\"o{\ss}enordnung 
$\frac{\sqrt[4]{\omega^3}}{\sqrt[4]{D-1}}$}\quad
\]
und also
\[
\mbox{$\frac{\mu}{N}$ von der Gr\"o{\ss}enordnung 
$\frac{1}{\sqrt[4]{\omega\left(D-1\right)}}$}
\]
ist. Die Beugung ist daher um so gr\"o{\ss}er, je n\"aher die
Quelle $S$ der Erdoberfl\"ache gelegen ist und je l\"anger die
entsendeten Wellen sind. Auf diese Weise wird die zun\"achst
staunenerregende Tatsache verst\"andlich, da{\ss} es mit Hilfe
der in der drahtlosen Telegraphie verwendeten
\textsc{Hertz}schen Wellen gelingt, vom europ\"aischen Kontinent
z.~B. bis nach Amerika zu telegraphieren.

Wenn man nicht den mittleren Wert der Reihe betrachten will,
welcher von einem Integral dargestellt wird, sondern den
wirklichen Wert dieser Reihe, so hat man eine Diskussion
durchzuf\"uhren, welche auf einem wohlbekannten \textsc{Abel}schen 
Satz\label{AbelscherSatz} beruht, und deren Resultate etwas
komplizierter, aber sonst ganz \"ahnlich den vorliegenden sind.

\vspace{5pt}

\selectlanguage{french}
\textbf{Note.} Je me suis aper\c{c}u que les derni\`eres
conclusions doivent \^etre modifi\'ees. Les formules
approch\'ees dont j'ai fait usage ne sont plus vraies lorsque
$n$ est tr\`es voisin de $\omega \varrho$. Elles doivent \^etre
alors remplac\'ees par d'autres, o\`u figure une transcendante
enti\`ere satisfaisant \`a l'\'equation diff\'erentielle
\[
y''= xy
\]

Mais les termes qui doivent \^etre ainsi modifi\'es sont en
petit nombre et j'avais cru d'abord que le r\'esultat final n'en
serait pas modifi\'e. Un examen plus approfondi m'a montr\'e
qu'il n'en est rien. La somme des termes modifi\'es est
comparable \`a celle des autres termes dont j'avais tenu compte
et qui est donn\'ee par la formule pr\'ec\'edente; il en
r\'esulte une compensation presque compl\`ete de sorte que la
valeur de $\mu$ donn\'ee par les formules d\'efinitives est
notablement plus petite que celle qui r\'esulterait des formules
pr\'ec\'edentes.

\selectlanguage{german}
\newpage

\begin{center}
\bigskip\bigskip\bigskip\label{vierter}
{\Large Vierter Vortrag}

\bigskip\bigskip\bigskip
{\Large \"UBER DIE \bigskip

REDUKTION DER ABELSCHEN INTEGRALE \bigskip 

UND DIE \bigskip

THEORIE DER FUCHSSCHEN FUNKTIONEN}
\end{center}

\newpage

Meine Herren! Ich habe die Absicht, Ihnen heute \"uber die
Reduktion der \textsc{Abel}schen Integrale im Zusammenhang mit
der Theorie der automorphen und insbesondere der
\textsc{Fuchs}schen Funktionen vorzutragen. \label{AbelscheIntegrale}

Ein System von \textsc{Abel}schen Funktionen von $p$ Variabeln
und $2p$ Perioden hei{\ss}t \textit{reduzibel}, wenn es sich auf
ein System von $q$ Variabeln und $2q$ Perioden $(q<p)$
zur\"uckf\"uhren l\"a{\ss}t. Hierbei ist es von vornherein von
Wichtigkeit, zwei F\"alle zu unterscheiden:

Im \textit{ersten} Falle soll es m\"oglich sein, das System $S$
\textsc{Abel}scher Funktionen von $p$ Variabeln durch eine
\textit{algebraische Kurve}\label{algebraischeCurve} $C$ vom
Geschlechte\label{CurveGeschlecht} $p$ zu erzeugen. Ebenso soll
das System $S'$ von $q$ Variabeln aus der Theorie eines
algebraischen Gebildes vom Geschlechte $q$ entspringen.

Dieser unser erste Fall ist aber bekanntlich nicht der
allgemeine; denn die Kurve $C$ h\"angt nur von $3p - 3$
Konstanten ab, w\"ahrend die allgemeinen \textsc{Abel}\-schen
Funktionen von $p$ Variabeln $\frac{p\left(p+1\right)}{2}$
Parameter enthalten. Dadurch ist der \textit{zweite} der beiden
F\"alle gegeben, die wir unterscheiden. In diesem Falle
n\"amlich soll mindestens eines der beiden Systeme $S$, $S'$
nicht aus der Theorie der algebraischen Gebilde entspringen.

In meinem heutigen Vortrag will ich mich durchaus auf den
erstgenannten Fall beschr\"anken. Aber auch dann mu{\ss} ich
noch zwei F\"alle unterscheiden. Wir kn\"upfen n\"amlich unsere
Betrachtungen an die beiden algebraischen Kurven $C$ und $C'$
an. Im Falle der Reduzibilit\"at besteht zwischen beiden eine
algebraische \textit{Korrespondenz}. \label{Korrespondenz} Die
Beschaffenheit derselben liegt der in Aussicht gestellten
Fallunterscheidung zugrunde.

Der erste Fall ist der folgende. Verm\"oge der Korrespondenz ist
jedem Punkte $M$ von $C$ ein und nur ein Punkt $M'$ von $C'$
zugeordnet, w\"ahrend umgekehrt jedem Punkte von $C'$ $n$ Punkte
von $C$ entsprechen. Ich nenne dann $n$ die
\textit{charakteristische Zahl} \label{charakteristische} der
Korrespondenz und sage, $C$ ist eine \textit{vielfache
Kurve}\label{vielfacheCurve} von $C'$.

Der eben genannte erste Fall ist aber nicht der allgemeine. Das
ist vielmehr der nun folgende zweite. Hier n\"amlich besteht die
Korrespondenz nicht zwischen einzelnen Punkten $M$ und $M'$,
sondern zwischen Systemen von Punkten $M_1,\dots,M_{\nu}$ von
$C$ mit den Koordinaten $x_1,y_1;\dots;x_{\nu},y_{\nu}$ und
${M_1}',\dots,{M_{\nu}}'$ von $C'$ mit den Koordinaten
${x_1}',{y_1}';\allowbreak\dotsc;\allowbreak {x_\nu}'{y_\nu}'$. 
Jedem System auf $C$ soll dabei ein und nur ein System auf $C'$ 
entsprechen, w\"ahrend umgekehrt einem System auf $C'$ im
allgemeinen mehrere Systeme auf $C$ zugeordnet sind. Ich sage
dann, $C$ ist eine \textit{pseudovielfache Kurve}
\label{pseudovielfacheCurve} von $C'$.

Im erstgenannten Falle sind $x'$ und $y'$ rationale Funktionen
von $x$ und $y$, w\"ahrend im zweiten nur geschlossen werden
kann, da{\ss} jede rationale und symmetrische Funktion der
$({x_1}'{y_1}',\dotsc,{x_\nu}'{y_\nu}')$ eine rationale Funktion
der $({x_1}{y_1},\allowbreak\dotsc, \allowbreak
{x_\nu}{y_\nu})$ ist. Es ist leicht zu sehen, da{\ss} jede Kurve
$C$, die eine vielfache von $C'$ ist, auch eine pseudovielfache der
Kurve $C'$ ist. Umgekehrt aber habe ich mehrere Beispiele bilden
k\"onnen daf\"ur, da{\ss} nicht jede pseudovielfache Kurve von
$C'$ auch eine vielfache von $C'$ ist. Ich will jedoch hier nicht
n\"aher darauf eingehen, zumal da sich meine folgenden Darlegungen 
durchaus an den \textsc{ersten Fall} anschlie{\ss}en werden.

Im Falle der Reduzibilit\"at unserer Integrale ist es m\"oglich,
ihre Periodentabelle\label{Periodentabelle} auf eine besondere
\textit{Normalform} zu bringen. Die folgenden beiden Beispiele
m\"ogen eine Anschauung von der Beschaffenheit derselben geben.

1) $q = 1$; $p = 3$. Die Periodentabelle kann auf die folgende
Form gebracht werden:
\[
\begin{array}{cccccc}  
  2 i \pi & 0 & 0 & h & \frac{2 i \pi}{\alpha} & 0 \\
  0 & 2 i \pi & 0 & \frac{2 i \pi}{\alpha} & a & b \\
  0 & 0 & 2 i \pi & 0 & b & c.
\end{array}
\]

2) $q = 2$; $p = 4$. Die normierten Perioden sind hier:
\[
\begin{array}{cccccccc} 
  2 i \pi & 0 & 0 & 0 & a & b & 0 & \frac{2 i \pi}{\alpha} \\
  0 & 2 i \pi & 0 & 0 & b & c & \frac{2 i \pi}{\alpha\beta} & 0 \\
  0 & 0 & 2 i \pi & 0 & 0 & \frac{2 i \pi}{\alpha\beta} & a' & b' \\
  0 & 0 & 0 & 2 i \pi & \frac{2 i \pi}{\alpha} & 0 & b' & c'.
\end{array}
\]

Die Zahlen $\alpha$, $\beta$ bedeuten in beiden Tabellen ganze
rationale Zahlen.

Ich definiere nun noch eine \textit{zweite charakteristische
Zahl $\kappa$}. Sie gibt die Ordnung der Thetafunktion von $q$
Variabeln an, in die eine Thetafunktion erster Ordnung von $p$
Variabeln im Falle der Reduzibilit\"at transformiert werden
kann. Im ersten Beispiel ist $\kappa = \alpha$, im zweiten
$\kappa = \alpha\beta$. \textit{Die beiden charakteristischen
Zahlen $n$ und $\kappa$ sind nun immer einander gleich.} Ich
habe zwei Beweise f\"ur diesen Satz gefunden, die ich jetzt in
ihren Grundz\"ugen auseinandersetzen will.

\textit{Erster Beweis}. Seien $M$ und $M'$ zwei
\textsc{Abel}sche Integrale erster, zweiter oder dritter Gattung
der Kurve $C$. Ich denke mir die zugeh\"orige \textsc{Riemann}sche 
Fl\"ache irgendwie l\"angs $2p$ von einem Punkte ausgehenden
nichtzerst\"uckenden R\"uckkehrschnitten kanonisch
aufgeschnitten.  Dann m\"ogen $M$ und $M'$ die folgenden
Perioden besitzen:
\begin{align*}
M &: x_1,x_2,\dots,x_{2p},\\
M' &: y_1,y_2,\dots,y_{2p}.
\end{align*}
Ich mu{\ss} nun eine charakteristische fundamentale
\textit{Bilinearform} definieren. Ich setze n\"amlich:
\[
F(x,y)=\int{MdM'}
\]
wo das Integral l\"angs der ganzen Kontur der Zerschneidung
erstreckt werden soll. Wenn $x$, $y$ Normalperioden sind, so
nimmt $F(x, y)$ die folgende Form an:
\[
F(x,y) = \sum\limits_{\kappa=1}^p
\left(x_{2\kappa-1}y_{2\kappa}-x_{2\kappa}y_{2\kappa-1}\right).
\]
Nehme ich an, es sei $M$ eines der reduziblen Integrale, dann
dr\"ucken sich seine $2p$ Perioden ganzzahlig und linear durch
nur $2q$ Perioden $\omega_1,\dots, \omega_{2q}$ aus. Ich habe
also dann:
\begin{align*}
&&&&&&x_{\kappa}=\sum\limits_{j=1}^{2q} m_{\kappa j}\omega_j
&&(\kappa=1,2,\dots,2p),
\end{align*}
wo die $m_{\kappa}$ ganze rationale Zahlen bedeuten. Wenn nun
$M$ und $M'$ Integrale erster Gattung sind, dann ist bekanntlich
\[
F(x,y) = 0
\]
Wenn man in diese Gleichung die Ausdr\"ucke der $x$ durch die
$\omega$ einsetzt, so bekommt man eine bilineare Gleichung
zwischen den $y$ und $\omega$, die in der folgenden Form
geschrieben werden kann:
\[
\sum\limits_{j=1}^{2q} H_j\omega_j = 0
\]
Seien nun $u_1,\dots,u_p$ $p$ linear unabh\"angige Integrale
erster Gattung von $C$. Dann k\"onnen wir setzen:

\begin{center}
\begin{math}
\begin{array}{llllllll}
U &= \mu_1 u_1 &+& \mu_2 u_2 &+& \dots &+ \mu_p u_p\\
U' &= {\mu_1}' u_1 &+& {\mu_2}' u_2 &+& \dots &+ {\mu_p}' u_p\\
\end{array}
\end{math}
\end{center}
Die vorl\"aufig noch unbestimmten Koeffizienten $\mu'$ sollen
nun so bestimmt werden, da{\ss} sie den $2q$ linearen
Gleichungen:
\begin{align*}
&&&&&&H_j=0
&&(j=1,2,\dots,2q)
\end{align*}
gen\"ugen. Wenn man dann noch beachtet, da{\ss} diese $2q$
Gleichungen nicht linear unabh\"angig sind, sondern da{\ss}
zwischen ihnen $q$ Relationen
\[
\sum H_j \omega_j =0
\]
bestehen, so ist leicht zu erkennen, da{\ss} auch $M_1$
reduzierbar ist, und da{\ss}, so wie $M$ einer Schar von $q$
reduziblen Integralen angeh\"ort, auch $M'$ ein Element einer
$(p-q)$fach unendlichen linearen Schar von reduziblen Integralen
ist. Doch dies nur nebenbei.

Ich bemerke nun, da{\ss} $H_j$ eine lineare Funktion der
$y_{\kappa}$ ist, soda{\ss} ich schreiben kann:
\begin{align*}
&&&&&&H_j=\sum\limits_{i=1}^{2p} h_{ij}y_{i}
&&(j=1,2,\dots,2q),
\end{align*}
wo die $h_{ij}$ ganze rationale Zahlen sind. Aus den
$m_{i\kappa}$ und den $h_{i\kappa}$ kann ich nun zwei Tabellen
von je $2q$ Kolonnen und $2p$ Zeilen bilden. Aus beiden kann ich
gewisse \mbox{$q$-reihige} Determinanten bilden. Ich bezeichne
die der $m$ mit $D$ und die aus denselben Zeilen der $h$
gebildete mit $D'$. Dann setze ich
\[
J=\sum{DD'}.
\]
$J$ ist nun in dem folgenden Sinne eine \textit{invariante}
\label{invariante} Zahl: Sie bleibt unge\"andert, wenn man
irgendeines der Periodensysteme $x$ oder $\omega$ durch ein
\"aquivalentes ersetzt. Dabei hei{\ss}en zwei Periodensysteme
\"aquivalent, wenn sie sich ganzzahlig und linear durcheinander
ausdr\"ucken lassen. Man kann nun einerseits beweisen, da{\ss}
\[
J=\kappa^2,
\]
andererseits aber, da{\ss}
\[
J=n^2.
\]
Daraus kann man folgern, da{\ss}
\[
\kappa=n.
\]
Das ist der erste Beweis. Der nun folgende

\textit{Zweite Beweis} ist wesentlich k\"urzer. Er beruht auf
dem Vergleich der zu $S$ und $S'$ geh\"origen Bilinearformen
$F(x,y)$ und $\Phi(\omega,\omega')$. Man hat n\"amlich
einerseits
\[
F(x,y) = n\Phi(\omega,\omega'),
\]
andererseits
\[
F(x,y) = \kappa\Phi(\omega,\omega').
\]

Daraus schlie{\ss}t man
\[
\kappa=n.
\]

Ich komme nun zum \textit{Zusammenhang der Reduktionstheorie
mit der Theorie der \textsc{Fuchs}schen Funktionen}.
\label{Fuchs}\label{Reduktionstheorie}

Bekanntlich definiert jede algebraische Kurve $C$ ein System von
\textsc{Fuchs}schen Funktionen. Nun kann man die Tatsache,
da{\ss} $C$ ein Vielfaches von $C'$ ist, auch folgenderma{\ss}en
ausdr\"ucken. Es ist immer auf mannigfache Weise m\"oglich, der
Kurve $C'$ eine Grenzkreisgruppe\label{Grenzkreisgruppe} $G'$
und $C$ eine ebensolche Gruppe $G$ zuzuordnen, soda{\ss} $G$
eine \textit{Untergruppe} von $G'$ ist. Ist im besonderen $C$
ein $n$-faches von $C'$, dann ist $G$ eine Untergruppe vom Index
$n$ von $G'$. Man erh\"alt daher einen Fundamentalbereich
\label{Fundamentalbereich} von $G$ dadurch, da{\ss} man $n$
geeignet gew\"ahlte Fundamentalbereiche von $G'$, die durch die
Operationen von $G'$ auseinander hervorgehen, aneinander lagert.
Das Polygon $P$ von $G$ erscheint dann in $n$ Polygone
$P'(\beta)$ eingeteilt, die einem Polygon $P'$ von $G'$ im Sinne
der nichteuklidischen Geometrie kongruent sind.\label{nichteuklid}

Ich bezeichne die \textit{Seiten} des Polygons $P'$ mit
$\gamma(\alpha)$ und die homologen Seiten von $P'(\beta)$ mit
$\gamma(\alpha,\beta)$. Die Seiten $\gamma(\alpha,\beta)$ liegen
entweder im Innern oder auf dem Rande von $P$. Ich nehme nun an,
die Seite $\gamma(\alpha')$ gehe aus $\gamma(\alpha)$ verm\"oge
einer Operation von $G'$ hervor. Wenn nun $\gamma(\alpha,\beta)$
auf dem Rande von $P$ liegt, dann gibt es eine weitere Seite
$\gamma(\alpha',\beta')$ auf diesem Rande, die mit
$\gamma(\alpha,\beta)$ verm\"oge einer Operation von $G$
konjugiert ist. Wenn jedoch $\gamma(\alpha,\beta)$ im Innern von
$P$ liegt, so existiert eine derartige von $\gamma(\alpha,\beta)$
verschiedene Seite nicht, sondern es fallen $\gamma(\alpha,\beta)$
und $\gamma(\alpha',\beta')$ zusammen und bilden die gemeinsame
Seite von $P'(\beta)$ und $P'(\beta')$. Aber wie dem auch sei,
jedenfalls entspricht jeder Seite $\gamma(\alpha)$ von $P'$ eine
Permutation der $n$ Ziffern $1, 2, \dotsc, n$.

Eine der eben durchgef\"uhrten ganz \"ahnliche Betrachtung
k\"onnen wir auch f\"ur die \textit{Ecken} von $P'$ anstellen.
So wie wir n\"amlich die Seiten in Paare zusammenfa{\ss}ten, so
k\"onnen wir die Ecken in Zyklen einteilen, so da{\ss} die Ecken
eines Zyklus auseinander durch die Operationen von $G'$
hervorgehen. Jedem solchen Zyklus kann wieder eine bestimmte
Vertauschung der $n$ Ziffern $1, 2, \dotsc, n$ zugeordnet werden,
die sich aus den den Seiten zugeordneten gewinnen l\"a{\ss}t.
Ich nehme nun an, es habe~$P$ $2N$~Seiten und $Q$ Eckenzyklen.
$2N'$ und $Q'$ sollen die gleiche Bedeutung f\"ur $P'$ haben.
Die einem Eckenzyklus von $P'$ entsprechende Permutation
l\"a{\ss}t sich in zyklische Permutationen zerlegen. Bei allen
Eckenzyklen m\"ogen dabei im ganzen $\lambda_i$ zyklische
Permutationen von gerade $i$ Ziffern vorkommen. Dann bestehen
die folgenden Relationen:
\begin{align*}
	2p &= N - Q + 1,\\
	2q &= N' - Q' + 1,\\
	Q+2p-2 &= n(Q'+2q-2),\\
	n(Q'-Q) &= 2(p-1)-2n(q-1),\\
	\sum{\lambda_i} &= Q,\\
	\sum{i\lambda_i} &= nQ'.
\end{align*}
Die bisher gegebenen allgemeinen Betrachtungen setzen uns nun
instand, eine Reihe sch\"oner und wichtiger \textit{S\"atze
\"uber die nichteuklidische Geometrie der Kreisbogenpolygone,
sowie \"uber die Geometrie der algebraischen
Kurven}\label{Geometrie} abzuleiten. Ich will im folgenden
einige Beispiele solcher S\"atze anf\"uhren, ohne mich des
n\"aheren auf Beweise einzulassen, deren Grundz\"uge \"ubrigens
im vorstehenden enthalten sind.

1) $p = 3, q = 2, n = 2, m = m' = 4$.

Mit $m$ und $m'$ sind dabei die Ordnungen der Kurven $C$ und
$C'$ bezeichnet. $C$ hat keinen Doppelpunkt, $C'$ hat einen
Doppelpunkt. \textit{Von den 28 Doppeltangenten von $C$ gehen
sechs durch einen Punkt au{\ss}erhalb der Kurve}.

2) $p = 4, q = 2, n = 2, m = 4, m' = 5$.

$C$ hat zwei Doppelpunkte, $C'$ nur einen. Setzt man die
Differentiale der reduziblen Integrale erster Gattung gleich
Null, so erh\"alt man ein \textit{Kegelschnittb\"uschel}, dessen
vier Basispunkte von den beiden Doppelpunkten von $C$ und zwei
weiteren Punkten derselben Kurve gebildet werden. Sechs dieser
Kegelschnitte ber\"uhren $C$ doppelt. Derjenige derselben, der
$C$ in einem Basispunkte ber\"uhrt, oskuliert daselbst.

3) $p = 2, q = 1, n = 2$.

Die Kurve $C$ ist ein \textit{Vielfaches von zwei verschiedenen}
Kurven $C'$ und $C''$. Es existiert eine \textsc{Fuchs}sche
Gruppe $G$, zu der man sowohl ein erstes Polygon $P_1$
konstruieren kann, das aus zwei Polygonen einer zu $C'$
geh\"origen Gruppe $G'$ besteht, als auch ein zweites Polygon
$P_2$, das aus zwei Polygonen einer zu $C''$ geh\"origen Gruppe
$G''$ besteht. $G$ ist also sowohl in $G'$ als in $G''$ als
Untergruppe vom Index $2$ enthalten. Die nebenstehende
schematische \textit{Figur} m\"oge zur Veranschaulichung der
Verh\"altnisse dienen. Die beiden eben erw\"ahnten
Fundamentalbereiche $P_1$ und $P_2$ von $G$ sind durch die
Polygone mit den Ecken $A$ bzw.~$C$ dargestellt. Jedes derselben
zerf\"allt in zwei Sechsecke, die bzw.~Fundamentalbereiche von
$G'$ oder $G''$ sind. Um die \"Aquivalenz von $P_1$ und $P_2$
besser hervortreten zu lassen, sind die Symmetriezentren der
erw\"ahnten Sechsecke mit den Seitenmitten verbunden, soda{\ss}
alle Polygone sich in leicht ersichtlicher Weise aus den so
entstehenden Vierecken aufbauen.

\begin{figure}[htb]
\begin{center}
\includegraphics*[scale=0.3]{fig5.png}
\end{center}
\end{figure}

Ich gehe nun zu den S\"atzen aus der Geometrie der algebraischen
Kurven \"uber, die uns dieses Beispiel lehrt. Wenn ich auf $C'$
einen Punkt $M'$ markiere, so entsprechen diesem zwei Punkte
$M_a$ und $M_b$ auf $C$. Jedem von diesen entspricht ein Punkt
von $C''$: ${M_a}''$, ${M_b}''$. Es entsprechen also im
allgemeinen jedem Punkte von $C'$ zwei Punkte von $C''$. Ebenso
kann man schlie{\ss}en, da{\ss} im allgemeinen jedem Punkte von
$C''$ zwei Punkte von $C'$ entsprechen. Die Korrespondenz $(C',
C)$ hat aber zwei Verzweigungspunkte ${M_1}'$, ${M_2}'$. Jedem
von ihnen entspricht also nur ein Punkt von $C$ und also auch
nur ein Punkt von $C''$: ${M_1}''$, ${M_2}''$. Ebenso hat die
Korrespondenz $(C'', C)$ zwei Verzweigungspunkte ${N_1}''$,
${N_2}''$. Jedem von ihnen ist nur ein Punkt von $C'$
zugeordnet: ${N_1}'$, ${N_2}'$. Wir k\"onnen dann den ersten
Satz, den wir anf\"uhren wollen, so aussprechen:

\textit{${N_1}'$ und ${N_2}'$ einerseits und ${M_1}''$ und
${M_2}''$ andererseits fallen zusammen.}

Ich gehe zum zweiten Satz \"uber, der sich ergibt, wenn man $C'$
und $C''$ als Kurven dritter Ordnung voraussetzt.

\textit{Ich kann in ${N_1}' = {N_2}'$ die Tangente an $C'$
ziehen. Ich verbinde ferner ${M_1}'$ und ${M_2}'$ durch eine
Sekante. Diese beiden Geraden schneiden sich auf $C'$. Ebenso
kann ich in ${M_1}'' = {M_2}''$ die Tangente an $C''$ ziehen und
mit der Sekante ${N_1}''{N_2}''$ zum Schnitt bringen. Der
Schnittpunkt liegt auf $C''$.}

Diese wenigen Beispiele lassen zur Gen\"uge erkennen, wie zahlreich 
die besonderen F\"alle sind.

\newpage

\begin{center}
\bigskip\bigskip\bigskip\label{fuenfter}
{\Large F\"unfter Vortrag}

\bigskip\bigskip\bigskip
{\Large \"UBER TRANSFINITE ZAHLEN}
\end{center}

\newpage

Meine Herren! Ich will heute \"uber den Begriff der transfiniten
Kardinalzahl \label{transfinite} vor Ihnen sprechen; und zwar
will ich zun\"achst von einem \textit{scheinbaren} Widerspruch
reden, den dieser Begriff enth\"alt. Dazu schicke ich folgendes
voraus: meiner Ansicht nach ist ein Gegenstand nur dann denkbar,
wenn er sich mit einer endlichen Anzahl von Worten definieren
l\"a{\ss}t. Einen Gegenstand, der in diesem Sinne endlich
definierbar ist, will ich zur Abk\"urzung einfach
``definierbar'' \label{definierbar} nennen. Demnach ist also ein
nicht definierbarer Gegenstand auch undenkbar. Desgleichen will
ich ein Gesetz ``aussagbar''\label{aussagbar} nennen, wenn es in
einer endlichen Anzahl von Worten ausgesagt werden kann.

Herr \textsc{Richard}\label{Richard} hat nun bewiesen, da{\ss}
die Gesamtheit der definierbaren Gegenst\"ande abz\"ahlbar ist,
d.~h.~da{\ss} die Kardinalzahl dieser Gesamtheit $\aleph_0$ ist.
Der Beweis ist ganz einfach: sei $\alpha$ die Anzahl der
W\"orter des W\"orterbuches, dann kann man mit $n$ W\"ortern
h\"ochstens $\alpha^n$ Gegenst\"ande definieren. L\"a{\ss}t man
nun $n$ \"uber alle Grenzen wachsen, so sieht man, da{\ss} man
nie \"uber eine abz\"ahlbare Gesamtheit hinauskommt. Die
M\"achtigkeit der Menge der denkbaren Gegenst\"ande w\"are also
$\aleph_0$. Herr \textsc{Schoenflies}\label{Schoenflies} hat
gegen diesen Beweis eingewandt, da{\ss} man mit einer einzigen
Definition mehrere, ja sogar unendlich viele Gegenst\"ande
definieren k\"onne. Als Beispiel f\"uhrt er die Definition der
konstanten Funktionen an, deren es offenbar unendlich viele
gibt. Dieser Einwand ist deshalb unzul\"assig, weil durch solche
Definitionen gar nicht die einzelnen Gegenst\"ande, sondern ihre
Gesamtheit, in unserem Beispiel also die \textit{Menge} der
konstanten Funktionen definiert wird, und diese ist ein einziger
Gegenstand. Der Einwand von Herrn \textsc{Schoenflies} ist also
nicht stichhaltig.

Nun hat bekanntlich \textsc{Cantor}\label{Cantor} bewiesen,
da{\ss} das Kontinuum nicht abz\"ahlbar ist; dies widerspricht
dem Beweise von \textsc{Richard}. Es fragt sich also, welcher
von beiden Beweisen richtig ist. Ich behaupte, sie sind beide
richtig, und der Widerspruch ist nur ein scheinbarer. Zur
Begr\"undung dieser Behauptung will ich einen neuen Beweis f\"ur
den \textsc{Cantor}schen Satz geben: Wir nehmen also an, es sei
eine Strecke $AB$ gegeben und ein Gesetz, durch welches jedem
Punkte der Strecke eine ganze Zahl zugeordnet wird. Wir wollen
der Einfachheit halber die Punkte durch die ihnen zugeordneten
Zahlen bezeichnen. Wir teilen nun unsere Strecke durch zwei
beliebige Punkte $A_1$ und $A_2$ in drei Teile, die wir als
Unterstrecken $1$.~Stufe bezeichnen; diese teilen wir wieder in
je drei Teile und erhalten Unterstrecken $2$.~Stufe; dieses
Verfahren denken wir uns ins Unendliche fortgesetzt, wobei die
L\"ange der Unterstrecken unter jede Grenze sinken soll. Der
Punkt~$1$ geh\"ort nun einer oder h\"ochstens, wenn er mit $A_1$
oder $A_2$ zusammenf\"allt, zweien der Unterstrecken erster
Stufe an, es gibt also sicher eine, der er nicht angeh\"ort. Auf
dieser suchen wir den Punkt mit der niedrigsten Nummer, die nun
mindestens $2$ sein mu{\ss}, auf. Unter den $3$ Unterstrecken
$2$.~Stufe, die zu derjenigen Strecke $1$.~Stufe geh\"oren, auf
der wir uns befinden, ist nun wieder mindestens eine, der der
zuletzt betrachtete Punkt nicht angeh\"ort. Auf dieser setzen
wir das Verfahren fort und erhalten so eine Folge von Strecken,
die folgende Eigenschaften hat: jede von ihnen ist in allen
vorhergehenden enthalten, und eine Strecke $n^\text{ter}$ Stufe
enth\"alt keinen der Punkte $1$ bis $n-1$. Aus der ersten
Eigenschaft folgt, da{\ss} es mindestens einen Punkt geben
mu{\ss}, der ihnen allen gemeinsam ist; aus der zweiten
Eigenschaft folgt aber, da{\ss} die Nummer dieses Punktes
gr\"o{\ss}er sein mu{\ss} als jede endliche Zahl, d.~h.~es kann
ihm keine Zahl zugeordnet werden.

Was haben wir nun zu diesem Beweise vorausgesetzt? Wir haben ein
Gesetz vorausgesetzt, das jedem Punkte der Strecke eine ganze
Zahl zuordnet. Dann konnten wir einen Punkt definieren, dem
keine ganze Zahl zugeordnet ist. In dieser Hinsicht
unterscheiden sich die verschiedenen Beweise dieses Satzes
nicht. Dazu mu{\ss}te aber das Gesetz zuerst feststehen. Nach
\textsc{Richard} m\"u{\ss}te anscheinend ein solches Gesetz
existieren, aber \textsc{Cantor} hat das Gegenteil bewiesen. Wie
kommen wir aus diesem Dilemma heraus? Fragen wir einmal nach der
Bedeutung des Wortes ``definierbar''. Wir nehmen die Tafel aller
endlichen S\"atze und streichen daraus alle diejenigen, die
keinen Punkt definieren. Die \"Ubrigbleibenden ordnen wir den
ganzen Zahlen zu. Wenn wir jetzt die Durchmusterung der Tafel
von neuem vornehmen, so wird es sich im allgemeinen zeigen,
da{\ss} wir jetzt einige S\"atze stehen lassen m\"ussen, die wir
vorher gestrichen haben. Denn die S\"atze, in welchen man von
dem Zuordnungsgesetz selbst sprach, hatten fr\"uher keine
Bedeutung, da die Punkte den ganzen Zahlen noch nicht zugeordnet
waren. Diese S\"atze haben jetzt eine Bedeutung, und m\"ussen in
unserer Tafel bleiben. W\"urden wir jetzt ein neues
Zuordnungsgesetz aufstellen, so w\"urde sich dieselbe
Schwierigkeit wiederholen und so ad infinitum. Hierin liegt aber
die L\"osung des scheinbaren Widerspruchs zwischen
\textsc{Cantor} und \textsc{Richard}. Sei $M_0$ die Menge der
ganzen Zahlen, $M_1$ die Menge der nach der ersten
Durchmusterung der Tafel aller endlichen S\"atze definierbaren
Punkte unserer Strecke, $G_1$ das Gesetz der Zuordnung zwischen
beiden Mengen. Durch dieses Gesetz kommt eine neue Menge $M_2$
von Punkten als definierbar hinzu. Zu $M_1+M_2$ geh\"ort aber
ein neues Gesetz $G_2$, dadurch entsteht eine neue Menge $M_3$
usw. \textsc{Richard}s Beweis lehrt nun, da{\ss}, wo ich auch
das Verfahren abbreche, immer ein Gesetz existiert, w\"ahrend
\textsc{Cantor} beweist, da{\ss} das Verfahren beliebig weit
fortgesetzt werden kann. Es besteht also kein Widerspruch
zwischen beiden.

Der Schein eines solchen r\"uhrt daher, da{\ss} dem
Zuordnungsgesetz von \textsc{Richard} eine Eigenschaft fehlt,
die ich mit einem von den englischen Philosophen entlehnten
Ausdruck als ``pr\"adikativ'' \label{praedikativ} bezeichne.
(Bei \textsc{Russell},\label{Russell} dem ich das Wort entlehne,
ist eine Definition zweier Begriffe $A$ und $A'$ nicht
pr\"adikativ, wenn $A$ in der Definition von $A'$ und umgekehrt
vorkommt.) Ich verstehe darunter folgendes: Jedes
Zuordnungsgesetz setzt eine bestimmte Klassifikation voraus. Ich
nenne nun eine Zuordnung pr\"adikativ, wenn die zugeh\"orige
Klassifikation pr\"adikativ ist. Eine Klassifikation aber nenne
ich pr\"adikativ, wenn sie durch Einf\"uhrung neuer Elemente
nicht ver\"andert wird. Dies ist aber bei der
\textsc{Richard}schen nicht der Fall, vielmehr \"andert die
Einf\"uhrung des Zuordnungsgesetzes die Einteilung der S\"atze
in solche, die eine Bedeutung haben, und solche, die keine
haben. Was hier mit dem Wort ``pr\"adikativ'' gemeint ist,
l\"a{\ss}t sich am besten an einem Beispiel illustrieren: wenn
ich eine Menge von Gegenst\"anden in eine Anzahl von Schachteln
einordnen soll, so kann zweierlei eintreten: entweder sind die
bereits eingeordneten Gegenst\"ande endg\"ultig an ihrem Platze,
oder ich mu{\ss} jedesmal, wenn ich einen neuen Gegenstand
einordne, die anderen oder wenigstens einen Teil von ihnen
wieder herausnehmen. Im ersten Falle nenne ich die
Klassifikation pr\"adikativ, im zweiten nicht. Ein gutes
Beispiel f\"ur eine nicht pr\"adikative Definition hat
\textsc{Russell} gegeben: $A$ sei die kleinste ganze Zahl, deren
Definition mehr als hundert deutsche Worte erfordert. $A$
mu{\ss} existieren, da man mit hundert Worten jedenfalls nur
eine endliche Anzahl von Zahlen definieren kann. Die Definition,
die wir eben von dieser Zahl gegeben haben, enth\"alt aber
weniger als hundert Worte. Und die Zahl $A$ ist also
\textit{definiert} als \textit{undefinierbar}.

\textsc{Zermelo}\label{Zermelo} hat nun gegen die Verwerfung der
nicht pr\"adikativen Definitionen den Einwand erhoben, da{\ss}
damit auch ein gro{\ss}er Teil der Mathematik hinf\"allig
w\"urde, z.~B.\ der Beweis f\"ur die Existenz einer Wurzel einer
algebraischen Gleichung.\label{Wurzeln}

Dieser Beweis lautet bekanntlich folgenderma{\ss}en:

Gegeben ist eine Gleichung $F(x) = 0$. Man beweist nun, da{\ss}
$\left|F(x)\right|$ ein Minimum haben mu{\ss}; sei $x_0$ einer
der Argumentwerte, f\"ur den das Minimum eintritt, also
\[
\left| F(x) \right| \geq \left| F(x_0) \right|
\]
Daraus folgt dann weiter, da{\ss} $F(x_0) = 0$ ist. Hier ist nun
die Definition von $F(x_0)$ nicht pr\"adikativ, denn dieser Wert
h\"angt ab von der Gesamtheit der Werte von $F(x)$, zu denen er
selbst geh\"ort.
 
Die Berechtigung dieses Einwandes kann ich nicht zugeben. Man
kann den Beweis so umformen, da{\ss} die nicht pr\"adikative
Definition daraus verschwindet. Ich betrachte zu diesem Zwecke
die Gesamtheit der Argumente von der Form $\frac{m+ni}{p}$, wo
$m$, $n$, $p$ ganze Zahlen sind. Dann kann ich dieselben
Schl\"usse wie vorher ziehen, aber der Argumentwert, f\"ur den
das Minimum von $\left|F(x)\right|$ eintritt, geh\"ort im
allgemeinen nicht zu den betrachteten. Dadurch ist der Zirkel im
Beweise vermieden. Man kann von jedem mathematischen Beweise
verlangen, da{\ss} die darin vorkommenden Definitionen usw.\
pr\"adikativ sind, sonst w\"are der Beweis nicht streng.

Wie steht es nun mit dem klassischen Beweise des
\textsc{Bernstein}schen\label{Bernstein} Theorems? Ist er
einwandfrei? Das Theorem sagt bekanntlich aus, da{\ss}, wenn
drei Mengen $A$, $B$, $C$ gegeben sind, wo $A$ in $B$ und $B$ in
$C$ enthalten ist, und wenn $A$ \"aquivalent $C$ ist, auch $A$
\"aquivalent $B$ sein mu{\ss}. Es handelt sich also auch hier um
ein Zuordnungsgesetz.  Wenn das erste Zuordnungsgesetz (zwischen
$A$ und $C$) pr\"adikativ ist, so zeigt der Beweis, da{\ss} es
auch ein pr\"adikatives Zuordnungsgesetz zwischen $A$ und $B$
geben mu{\ss}.

Was nun die zweite transfinite Kardinalzahl $\aleph_1$ betrifft,
so bin ich nicht ganz \"uberzeugt, da{\ss} sie existiert. Man
gelangt zu ihr durch Betrachtung der Gesamtheit der
Ordnungszahlen von der M\"achtigkeit $\aleph_0$; es ist klar,
da{\ss} diese Gesamtheit von h\"oherer M\"achtigkeit sein
mu{\ss}. Es fragt sich aber, ob sie abgeschlossen ist, ob wir
also von ihrer M\"achtigkeit ohne Widerspruch sprechen d\"urfen.
Ein aktual Unendliches gibt es jedenfalls nicht.

Was haben wir von dem ber\"uhmten \textit{Kontinuumproblem}
\label{Kontinuum} zu halten? Kann man die Punkte des Raumes
wohlordnen? Was meinen wir damit? Es sind hier zwei F\"alle
m\"oglich: entweder behauptet man, da{\ss} das Gesetz der
Wohlordnung endlich aussagbar ist, dann ist diese Behauptung
nicht bewiesen; auch Herr \textsc{Zermelo} erhebt wohl nicht den
Anspruch, eine solche Behauptung bewiesen zu haben. Oder aber
wir lassen auch die M\"oglichkeit zu, da{\ss} das Gesetz nicht
endlich aussagbar ist. Dann kann ich mit dieser Aussage keinen
Sinn mehr verbinden, das sind f\"ur mich nur leere Worte. Hier
liegt die Schwierigkeit. Und das ist wohl auch die Ursache f\"ur
den Streit \"uber den fast genialen Satz \textsc{Zermelos}.
Dieser Streit ist sehr merkw\"urdig:  die einen verwerfen das
Auswahlpostulat,\label{Auswahlpostulat} halten aber den Beweis
f\"ur richtig, die anderen nehmen das Auswahlpostulat an,
erkennen aber den Beweis nicht an.

Doch ich k\"onnte noch manche Stunde dar\"uber sprechen, ohne
die Frage zu l\"osen.

\newpage

\begin{center}
\bigskip\bigskip\bigskip\label{sechster}
{\Large Sechster Vortrag}

\selectlanguage{french}
\bigskip\bigskip\bigskip
{\Large LA M\'ECANIQUE NOUVELLE}
\end{center}

\newpage

Mesdames, messieurs!

Aujourd'hui, je suis oblig\'e de parler fran\c{c}ais, et il faut
que je m'en excuse. Il est vrai que dans mes pr\'ec\'edentes
conf\'erences je me suis exprim\'e en allemand, en un tr\`es
mauvais allemand: parler les langues \'etrang\`eres, voyez-vous,
c'est vouloir marcher lorsqu'on est boiteux; il est n\'ecessaire
d'avoir des b\'equilles; mes b\'equilles, c'\'etaient jusqu'ici
les formules math\'ematiques et vous ne sauriez vous imaginer
quel appui elles sont pour un orateur qui ne se sent pas tr\`es
solide. --- Dans la conf\'erence de ce soir, je ne veux pas user
de formules, je suis sans b\'equilles, et c'est pourquoi je dois
parler fran\c{c}ais.

En ce monde, vous le savez, il n'est rien de d\'efinitif, rien
d'immuable; les empires les plus puissants, les plus solides, ne
sont pas \'eternels: c'est l\`a un th\`eme que les orateurs
sacr\'es se sont plu bien souvent \`a d\'evelopper. --- Les
th\'eories scientifiques sont comme les empires, elles ne sont
pas assur\'ees du lendemain. Si l'une d'elles semblait \`a
l'abri des injures du temps, c'\'etait, certes, la m\'ecanique
newtonienne: elle paraissait incontest\'ee, c'\'etait un
monument imp\'erissable; et voil\`a qu'\`a son tour, je ne dirai
pas que le monument est par terre, ce serait pr\'ematur\'e, mais
en tout cas il est fortement \'ebranl\'e. Il est soumis aux
attaques de grands d\'emolisseurs: vous en avez un parmi vous,
M. Max \textsc{Abraham},\label{Abraham} un autre est le
physicien hollandais M. \textsc{Lorentz}.\label{Lorentz} Je
voudrais, en quelques mots, vous parler des ruines de l'ancien
\'edifice et du nouveau b\^atiment que l'on veut \'elever \`a
leur place. ---

Tout d'abord qu'est-ce qui caract\'erisait l'ancienne
m\'ecanique? C'\'etait ce fait tr\`es simple: je consid\`ere un
corps en repos, je lui communique une impulsion, c'est \`a dire
je fais agir sur lui, pendant un temps donn\'e une force
donn\'ee; le corps se met en mouvement, acquiert une certaine
vitesse; le corps \'etant anim\'e de cette vitesse, faisons agir
encore la m\^eme force pendant le m\^eme temps, la vitesse sera
doubl\'ee; si nous continuons encore, la vitesse sera tripl\'ee
apr\`es que nous aurons une troisi\`eme fois donn\'e une impulsion
identique. Recommen\c{c}ons ainsi un nombre suffisant de fois,
le corps finira par acqu\'erir une vitesse tr\`es grande, qui
pourra d\'epasser toute limite, une vitesse infinie.

Dans la nouvelle m\'ecanique, au contraire, on suppose qu'il est
impossible de communiquer \`a un corps partant du repos une
vitesse sup\'erieure \`a celle de la lumi\`ere. Que se
passe-t-il? Je consid\`ere le m\^eme corps au repos; je lui
donne une premi\`ere impulsion, la m\^eme que pr\'ec\'edemment,
il prendra la m\^eme vitesse; renouvelons une seconde fois cette
impulsion, la vitesse va encore augmenter, mais elle ne sera
plus doubl\'ee; une troisi\`eme impulsion produira un effet
analogue, la vitesse augmente mais de moins en moins, le corps
oppose une r\'esistance qui devient de plus en plus grande.
Cette r\'esistance, c'est l'inertie, c'est ce qu'on appelle
commun\'ement la masse; tout ce passe alors dans cette nouvelle
m\'ecanique comme si la masse n'\'etait pas constante, mais
croissait avec la vitesse. Nous pouvons repr\'esenter
graphiquement les ph\'enom\`enes: dans l'ancienne m\'ecanique,
le corps prend apr\`es la premi\`ere impulsion une vitesse
represent\'ee par le segment $\overline{O\nu_1}$; apr\`es la
deuxi\`eme impulsion $\overline{O\nu_1}$ s'accro{\^\i}t d'un
segment $\overline{\nu_1\nu_2}$ qui lui est \'egal, \`a chaque
nouvelle impulsion, la vitesse s'accro{\^\i}t de la m\^eme
quantit\'e, le segment qui la repr\'esente s'accro{\^\i}t d'une
longueur constante;  dans la nouvelle m\'ecanique, le segment
vitesse s'accro{\^\i}t de segments $\overline{\nu_1'\nu_2}$,
$\overline{\nu_2'\nu_3'}, \ldots$ qui sont de plus en plus
petits et tels que nous ne pouvons pas d\'epasser une certaine
limite, la vitesse de la lumi\`ere.

\begin{figure}[htb]
\begin{center}
\includegraphics*[scale=0.3]{fig6.png}
\end{center}
\end{figure}

Comment a-t-on \'et\'e conduit \`a de telles conclusions? A-t-on
fait des ex\-p\'eriences directes? Les divergences ne se
produiront que pour les corps anim\'es de grandes vitesses;
c'est alors seulement que les diff\'erences signal\'ees
deviennent sensibles. Mais, qu'est-ce qu'une tr\`es grande
vitesse? Est-ce celle d'une automobile qui fait 100 kilom\`etres
\`a l'heure; on s'extasie dans la rue sur une telle rapidit\'e;
\`a notre point de vue, c'est pourtant bien peu, une vitesse
d'escargot. L'astronomie nous donne mieux:
Mercure,\label{Mercur1} le plus rapide des corps c\'elestes
parcourt lui aussi 100 kilom\`etres environ, non plus \`a
l'heure mais \`a la seconde: pourtant, cela ne suffit pas
encore, de telles vitesses sont trop faibles pour r\'ev\'eler
les diff\'erences que nous voudrions observer. Je ne parle pas
de nos boulets de canon, ils sont plus rapides que les
automobiles, mais beaucoup plus lents que Mercure; vous savez
cependant qu'on a d\'ecouvert une artillerie dont les
projectiles sont beaucoup plus vite: je veux parler du radium
qui envoie dans tous les sens de l'\'energie, des projectiles;
la rapidit\'e du tir est bien plus grande, la vitesse initiale
est de $100\,000$ kilom\`etres par seconde, le tiers de la
vitesse de la lumi\`ere; le calibre des projectiles, leur poids,
sont, il est vrai, bien plus faibles et nous ne devons pas
compter sur cette artillerie pour augmenter la puissance
militaire de nos arm\'ees. Peut-on exp\'erimenter sur ces
projectiles? De telles exp\'eriences ont \'et\'e effectivement
tent\'ees; sous l'influence d'un champ \'electrique, d'un champ
magn\'etique il se produit une d\'eviation qui permet de se
rendre compte de l'inertie et de la mesurer. On a constat\'e
ainsi que la masse d\'epend de la vitesse et \'enoncer cette
loi: L'inertie d'un corps cro{\^\i}t avec sa vitesse qui reste
inf\'erieure \`a celle de la lumi\`ere, $300\,000$ kilom\`etres
par seconde.

Je passe maintenant au deuxi\`eme principe, le principe de
relativit\'e.  Je suppose un observateur qui se d\'eplace vers
la droite; tout se passe pour lui comme s'il \'etait au repos,
les objets qui l'entourent se d\'epla\c{c}ant vers la gauche:
aucun moyen ne permet de savoir si les objets se d\'eplacent
r\'eellement, si l'observateur est immobile ou en mouvement. On
l'enseigne dans tous les cours de m\'ecanique, le passager sur
le bateau croit voir le rivage du fleuve se d\'eplacer, tandis
qu'il est doucement entra{\^\i}n\'e par le mouvement du navire.
Examin\'ee de plus pr\`es, cette simple notion acquiert une
importance capitale; on n'a aucun moyen de trancher la question,
aucune experience ne peut mettre en defaut le principe: il n'y a
pas d'espace absolu, tous les d\'eplacements que nous pouvons
observer sont des deplacements relatifs. Ces considerations bien
famili\`eres aux philosophes, j'ai eu quelquefois l'occasion de
les exprimer: j'en ai m\^eme recueilli une publicit\'e dont je
me serais volontiers pass\'e, tous les journaux r\'eactionnaires
fran\c{c}ais m'ont fait d\'emontrer que le soleil tournait
autour de la terre; dans le fameux proc\`es entre l'Inquisition et
Galil\'ee,\label{Galilei} Galil\'ee aurait eu tous les torts.

Revenons \`a l'ancienne m\'ecanique: elle admettait le principe de
relativit\'e; au lieu d'\^etre fond\'ees sur des exp\'eriences,
ses lois \'etaient d\'eduites de ce principe fondamental. Ces
considerations suffisaient pour les ph\'enom\`enes purement
m\'ecaniques, mais cela n'allait plus pour d'importantes parties
de la physique, l'optique par exemple. On consid\'erait comme
absolue la vitesse de la lumi\`ere relativement \`a l'\'ether:
cette vitesse pouvait \^etre mesur\'ee, on avait th\'eoriquement
le moyen de comparer le d\'eplacement d'un mobile \`a un
deplacement absolu, le moyen de d\'ecider si oui ou non un corps
\'etait en mouvement absolu.

Des exp\'eriences d\'elicates, des appareils extr\^emement
pr\'ecis, que je ne d\'ecrirai pas devant vous, ont permis
d'essayer la r\'ealisation pratique d'une pareille comparaison:
le r\'esultat a \'et\'e nul. Le principe de r\'elativite n'admet
aucune restriction dans la nouvelle m\'ecanique; il a, si j'ose
ainsi dire, une valeur absolue.

Pour comprendre le r\^ole que joue le principe de relativit\'e
dans la Nouvelle M\'ecanique, nous sommes d'abord amen\'es \`a
parler du temps apparent,\label{tempsapparent} une invention
fort ing\'enieuse du physicien \textsc{Lorentz}. Nous supposons
deux observateurs l'un $A$ \`a Paris, l'autre $B$ \`a Berlin.
$A$ et $B$ ont des chronom\`etres identiques et veulent les
r\'egler: mais ce sont des observateurs m\'eticuleux comme il
n'y en a gu\`ere; ils exigent dans leur r\'eglage une
extraordinaire exactitude: ce sera, par exemple, non une
seconde, mais un milliardi\`eme de seconde. Comment pourront-ils
faire? De Paris \`a Berlin, $A$ envoie un signal
t\'el\'egraphique, avec un sans-fil, si vous voulez, pour \^etre
tout \`a fait moderne.  $B$ note le moment de la r\'eception et
ce sera pour les deux chronom\`etres l'origine des temps. Mais
le signal emploie un certain temps pour aller de Paris \`a
Berlin, il ne va qu'avec la vitesse de la lumi\`ere; la montre
de $B$ serait donc en retard; $B$ est trop intelligent pour ne
point s'en rendre compte; il va rem\'edier a cet inconvenient.
La chose semble bien simple: on cro{\^\i}se les signaux, $A$
re\c{c}oit et $B$ envoie, on prend la moyenne des corrections
ainsi faites, on a l'heure exacte. Mais cela est-il bien
certain? Nous supposons que de $A$ \`a $B$ le signal emploie le
m\^eme temps que pour aller de $B$ \`a $A$. Or $A$ et $B$ sont
emport\'es dans le mouvement de la terre par rapport \`a
l'\'ether, v\'ehicule des ondes \'electriques. Quand $A$ a
envoy\'e son signal il fuit devant lui, $B$ s'\'eloigne de
m\^eme, le temps employ\'e sera plus long que si les deux
observateurs \'etaient au repos; si au contraire c'est $B$ qui
envoie, $A$ qui re\c{c}oit, le temps est plus court parce que
$A$ va au devant des signaux; il leur est absolument impossible
de savoir si leurs chronom\`etres marquent ou non la m\^eme
heure. Quelle que soit la m\'ethode employ\'ee les
inconv\'enients restent les m\^emes l'observation d'un
ph\'enom\`ene astronomique, une m\'ethode optique quelconque se
heurtent aux m\^emes difficult\'es, $B$ ne pourra jamais
conna{\^\i}tre qu'une diff\'erence apparente de temps, qu'une
esp\`ece d'heure locale. Le principe de relativit\'e s'applique
int\'egralement.

Dans l'ancienne m\'ecanique pourtant, on d\'emontrait avec ce
principe toutes les lois fondamentales. On pourrait \^etre
tent\'e de reprendre les raisonnements classiques et de
raisonner comme il suit? Soit encore deux observateurs, $A$ et
$B$ pour les nommer comme on nomme toujours deux observateurs en
math\'emati\-ques; supposons les en mouvement, s'\'eloignant l'un
de l'autre; aucun d'eux ne peut d\'epasser la vitesse de la
lumi\`ere; par exemple $B$ sera anim\'e de $200\,000$
kilom\'etres vers la droite, $A$ de $200\,000$ vers la gauche.
$A$ peut se croire au repos et la vitesse apparente de $B$ sera,
pour lui, $400\,000$ kilom\`etres.  Si $A$ connait la
m\'ecanique nouvelle il se dira: $B$ a une vitesse qu'il ne peut
atteindre, c'est donc que moi aussi je suis en mouvement. Il
semble qu'il pourrait d\'ecider de sa situation absolue. Mais il
faudrait qu'il puisse observer le mouvement de $B$ lui meme;
pour faire cette observation $A$ et $B$ commencent par r\`egler
leurs montres, puis $B$ envoie \`a $A$ des t\'el\'egrammes pour
lui indiquer ses positions successives; en les r\'eunissant $A$
peut se rendre compte du mouvement de $B$ et tracer la courbe de
ce mouvement. Or les signaux se propagent avec la vitesse de la
lumi\`ere; les montres qui marquent le temps apparent varient
\`a chaque instant et tout se passera comme si la montre de $B$
avan\c{c}ait. $B$ croira aller beaucoup moins vite et la vitesse
apparente qu'il aura relativement \`a $A$ ne d\'epassera pas la
limite qu'elle ne doit pas atteindre. Rien ne pourra r\'ev\'eler
\`a $A$ s'il est en mouvement ou en repos absolu.

Il faut encore faire une troisi\`eme hypoth\`ese beaucoup plus
surprenante, beaucoup plus difficile \`a admettre, qui g\^ene
beaucoup nos habitudes actuelles. Un corps en mouvement de
translation subit une d\'eformation dans le sens m\^eme o\`u il
se d\'eplace; une sph\`ere, par exemple, devient comme une
esp\`ece d'ellipso{\"\i}de aplati dont le petit axe serait
parall\`ele \`a la translation. Si l'on ne s'aper\c{c}oit pas
tous les jours d'une transformation pareille c'est qu'elle est
d'une petitesse qui la rend presque imperceptible. La terre,
emport\'ee dans sa r\'evolution sur son orbite se d\'eforme
environ de $\frac{1}{200\,000\,000}$: pour observer un pareil
ph\'enom\`ene il faudrait des instruments de mesure d'une
pr\'ecision extr\^eme, mais leur pr\'ecision serait infinie
qu'on n'en serait pas plus avanc\'e car emport\'es eux aussi
dans le mouvement ils subiront la m\^eme transformation. On ne
s'apercevra de rien; le m\`etre que l'on pourrait employer
deviendra plus court comme la longueur qu'on mesure. On ne peut
savoir quelque chose qu'en comparant \`a la vitesse de la
lumi\`ere la longueur de l'un de ces corps. Ce sont l\`a de
delicates experiences, r\'ealis\'ees par \textsc{Michelson}
\label{Michelson} et dont je ne vous exposerai pas le d\'etail;
elles ont donn\'e des resultats tout \`a fait remarquables;
quelqu'\'etranges qu'il nous paraissent, il faut admettre que la
troisi\`eme hypoth\`ese est parfaitement v\'erifi\'ee.

Telles sont les bases de la nouvelle m\'ecanique, avec l'appui
de ces hypoth\`eses on trouve qu'elle est compatible avec le
principe de relativit\'e.

Mais il faut la rattacher alors \`a une conception nouvelle de la
mati\`ere.

Pour le physicien moderne, l'atome n'est plus l'\'el\'ement
simple; il est devenu un v\'eritable univers dans lequel des
milliers de plan\`etes gravitent autour de soleils minuscules.
Soleils et plan\`etes sont ici des particules
\textit{\'electris\'ees} soit n\'egativement soit positivement;
le physicien les appelle \textit{\'electrons}\label{Elektronen}
et b\^atit le monde avec elles. D'aucuns se repr\'esentent
l'atome neutre comme une masse centrale positive autour de
laquelle circulent un grand nombre d'\'electrons charg\'es
n\'egativement, dont la masse \'electrique totale est \'egale en
grandeur \`a celle du noyau central.

Cette conception de la mati\`ere permet de rendre compte
ais\'ement de l'aug\-mentation de la masse d'un corps avec sa
vitesse, dont nous avons fait un des caract\`eres de la
m\'ecanique nouvelle. Un corps quelconque n'\'etant qu'un
assemblage d'\'electrons, il nous suffira de le montrer sur ces
derniers. Remarquons, \`a cet effet, qu'un \'electron isol\'e se
d\'epla\c{c}ant \`a travers l'\'ether engendre un courant
\'electrique, c'est-\`a-dire un champ \'electromagn\'etique. Ce
champ correspond \`a une certaine quantit\'e d'\'energie
localis\'ee non dans l'\'electron, mais dans l'\'ether. Une
variation en grandeur ou en direction de la vitesse de
l'\'electron modifie le champ et se traduit par une variation de
l'\'energie \'electromagn\'etique de l'\'ether. Alors que dans
la m\'ecanique newtonienne la d\'epense d'\'energie n'est due
qu'\`a l'inertie du corps en mouvement, ici une partie de cette
d\'epense est due \`a ce que l'on peut appeler l'inertie de
l'\'ether relativement aux forces \'electromagn\'etiques.
L'inertie de l'\'ether augmente avec la vitesse et sa limite
devient infinie lorsque la vitesse tend vers la vitesse de la
lumi\`ere. La masse apparente de l'\'electron augmente donc avec
la vitesse; les exp\'eriences de \textsc{Kaufmann}
\label{Kaufmann} montrent que la masse r\'eelle constante de
l'\'electron est n\'egligeable par rapport \`a la masse
apparente et peut \^etre consid\'er\'ee comme nulle.

Dans cette nouvelle conception, la masse constante de la
mati\`ere a disparu. L'\'ether seul, et non plus la mati\`ere,
est inerte. Seul l'\'ether oppose une r\'esistance au mouvement,
si bien que l'on pourrait dire: il n'y a pas de mati\`ere, il
n'y a que des trous dans l'\'ether. Pour les mouvements
stationnaires ou quasi-stationnaires, la m\'ecanique nouvelle ne
diff\`ere pas --- au degr\'e d'approximation de nos mesures
pr\`es --- de la m\'ecanique newtonienne, avec cette seule
diff\'erence que la masse n'est plus ind\'ependante ni de la
vitesse, ni de l'angle que fait cette vitesse avec la direction
de la force acc\'el\'eratrice. Si par contre la vitesse a une
acc\'el\'eration consid\'erable, dans le cas, par ex.,
d'oscillations tr\`es rapides, il y a production d'ondes
hertziennes repr\'esentant une perte d'\'energie de l'\'electron
entra{\^\i}nant l'amortissement de son mouvement. Ainsi, dans la
t\'el\'egraphie sans fil, les ondes \'emises sont dues aux
oscillations des \'electrons dans la d\'echarge oscillante.

Des vibrations analogues ont lieu dans une flamme et de m\^eme
encore dans un solide incandescent. Pour \textsc{Lorentz}, il
circule \`a l'int\'erieur d'un corps incandescent un nombre
consid\'erable d'\'electrons qui, ne pouvant pas en sortir,
volent dans tous les sens et se r\'efl\'echissent sur sa
surface. On pourrait les comparer \`a une nu\'ee de moucherons
enferm\'es dans un bocal et venant frapper de leurs ailes les
parois de leur prison. Plus la temperature est \'elev\'ee, plus
le mouvement de ces \'electrons est rapide et plus les chocs
mutuels et les r\'eflexions sur la paroi sont nombreuses. A
chaque choc et \`a chaque r\'eflexion une onde
\'electromagn\'etique\label{ondeelectrique} est \'emise et c'est
la perception de ces ondes qui nous fait para{\^\i}tre le corps
incandescent.

Le mouvement des \'electrons est presque tangible, dans un tube
de \textsc{Crookes}.\label{Crookes} Il s'y produit un
v\'eritable bombardement d'\'electrons partant de la cathode.
Ces rayons cathodiques frappent violemment l'anticathode et s'y
r\'efl\'echissent en partie donnant ainsi naissance \`a un
\'ebranlement \'electromagn\'etique que plusieurs physiciens
identifient avec les rayons \textsc{R\"ontgen}.\label{Roentgen}

Il nous reste en terminant \`a examiner les relations de la
m\'ecanique nouvelle avec l'astronomie. La notion de masse
constante d'un corps s'\'evanouissant, que deviendra la loi de
\textsc{Newton}? Elle ne pourra subsister que pour des corps en
repos. De plus il faudra tenir compte du fait que l'attraction
n'est pas instantan\'ee. On peut donc se demander avec raison si
la m\'ecanique nouvelle ne va r\'eussir qu'\`a compliquer
l'astronomie sans obtenir une approximation sup\'erieure a celle
que nous donne la m\'ecanique c\'eleste classique. Mr.
\textsc{Lorentz} a abord\'e la question. Partant de la loi de
\textsc{Newton} suppos\'ee vraie pour deux corps \'electris\'es
au repos, il calcule l'action \'electrodynamique des courants
engendr\'es par ces corps en mouvement; il obtient ainsi une
nouvelle loi d'attraction contenant les vitesses des deux corps
comme param\`etres. Avant d'examiner comment cette loi rend
compte des ph\'enom\`enes astronomiques, remarquons encore que
l'acc\'el\'eration des corps c\'elestes a comme cons\'equence un
rayonnement \'electromagn\'etique, donc une dissipation de
l'\'energie se faisant ressentir en retour par un amortissement
de leur vitesse. A la longue, les plan\`etes finiront donc par
tomber sur le soleil. Mais cette perspective ne peut gu\`ere
nous effrayer, la catastrophe ne pouvant arriver que dans
quelques millions de milliards de si\`ecles. Revenant maintenant
\`a la loi d'attraction, nous voyons ais\'ement que la
diff\'erence entre les deux m\'ecaniques sera d'autant plus
grande que la vitesse des plan\`etes sera plus grande. S'il y a
une diff\'erence appr\'eciable, ce sera donc pour
Mercure\label{Mercur2} qu'elle sera la plus grande, Mercure
ayant de toutes les plan\`etes la plus grande vitesse. Or il
arrive justement que Mercure pr\'esente une anomalie non encore
expliqu\'ee: le mouvement de son p\'erih\'elie est plus rapide
que le mouvement calcul\'e par la theorie classique.
L'acc\'el\'eration est de $38''$ trop grande.
\textsc{Leverrier}\label{Leverrier} attribua cette anomalie \`a
une plan\`ete non encore d\'ecouverte et un astronome amateur
crut observer son passage au soleil. Depuis lors plus personne
ne l'a vue et il est malheureusement certain que cette plan\`ete
aper\c{c}ue n'\'etait qu'un oiseau. Or la m\'ecanique nouvelle
rend bien compte du sens de l'erreur relative \`a Mercure, mais
elle laisse cependant encore une marge de $32''$ entre elle et
l'observation. Elle ne suffit donc pas \`a ramener la
concordance dans la th\'eorie de Mercure. Si ce r\'esultat n'est
gu\`ere d\'ecisif en faveur de la m\'ecanique nouvelle, il est
encore moins d\'efavorable \`a son acceptation puisque le sens
dans lequel elle corrige l'\'ecart de la th\'eorie classique est
le bon. La th\'eorie des autres plan\`etes n'est pas
sensiblement modifi\'e dans la nouvelle th\'eorie et les
r\'esultats co{\"\i}ncident \`a l'approximation des mesures
pr\`es \`a ceux de la th\'eorie classique.

Pour conclure, il serait pr\'ematur\'e, je crois, malgr\'e la
grande valeur des arguments et des faits \'erig\'es contre elle,
de regarder la m\'ecanique classique comme d\'efinitivement
condamn\'ee. Quoiqu'il en soit d'ailleurs, elle restera la
m\'ecanique des vitesses tr\`es petites par rapport \`a celle de
la lumi\`ere, la m\'ecanique donc de notre vie pratique et de
notre technique terrestre. Si cependant, dans quelques ann\'ees
sa rivale triomphe, je me permettrai de vous signaler un
\'ecueil p\'edagogique que n'\'eviteront pas nombre de ma{\^\i}tres,
en France, tout au moins. Ces ma{\^\i}tres n'auront rien de plus
press\'e, en enseignant la m\'ecanique \'el\'ementaire \`a leurs
\'el\`eves, que de leur apprendre que cette m\'ecanique l\`a a
fait son temps, qu'une m\'ecanique nouvelle o\`u les notions de
masse et de temps ont une toute autre valeur la remplace; ils
regarderont de haut cette m\'ecanique p\'erim\'ee que les
programmes les forcent \`a enseigner et feront sentir \`a leurs
\'el\`eves le m\'epris qu'ils lui portent. Je crois bien
cependant que cette m\'ecanique classique d\'edaign\'ee sera
aussi n\'ecessaire que maintenant et que celui qui ne la
conna{\^\i}tra pas \`a fond ne pourra comprendre la m\'ecanique
nouvelle.

\newpage

\selectlanguage{german}
\section*{Personen- und Sachregister}

$ $

\textbf{A}\\
\\
\textsc{Abel}sche Integrale \pageref{AbelscheIntegrale} ff. \\
---scher Satz \pageref{AbelscherSatz} \\
\textsc{Abraham} \pageref{Abraham} \\
aussagbar \pageref{aussagbar} ff. \\
Auswahlpostulat \pageref{Auswahlpostulat} 

\bigskip
\textbf{B} \\
\\
\textsc{Bernstein} \pageref{Bernstein} \\
\textsc{Bessel}sche Funktionen \pageref{Bessel} \\
Breite, kritische geographische \pageref{kritischegeographische} 

\bigskip
\textbf{C} \\
\\
\textsc{Cantor} \pageref{Cantor} ff. \\
Colatitude \pageref{Colatitude} \\
\textsc{Crooke}sche R\"ohren \pageref{Crookes} \\
Curve, algebraische \pageref{algebraischeCurve} \\
--- Geometrie auf einer solchen \pageref{Geometrie} ff. \\
--- ihr Geschlecht \pageref{CurveGeschlecht} \\
--- vielfache einer anderen \pageref{vielfacheCurve} \\
--- pseudovielfache einer anderen \pageref{pseudovielfacheCurve} 

\bigskip
\textbf{D} \\
\\
definierbar \pageref{definierbar} ff. 

\bigskip
\textbf{E} \\
\\
Elektrizit\"at, Dichte der \pageref{Dichte} \\
onde \'electrique \pageref{ondeelectrique} \\
elektrische Verschiebung \pageref{Versch} \\
Elektronen \pageref{Elektronen} ff. 

\bigskip
\textbf{F} \\
\\
Flutproblem \pageref{Flutproblem} \\
\textsc{Fourier}sches Integraltheorem \pageref{FourierIntegral} \\
---sche Reihe \pageref{FourierReihe} \\
\textsc{Fredholm}sche Methode \pageref{Fredholm1} ff., 
  \pageref{Fredholm2}, \pageref{Fredholm3}, 15, 26 \\
\textsc{Fuchs}sche Funktionen \pageref{Fuchs} \\
Fundamentalbereich \pageref{Fundamentalbereich} 

\bigskip
\textbf{G} \\
\\
\textsc{Galilei} \pageref{Galilei} \\
\textsc{Green}sche Funktion \pageref{Green} \\
Grenzkreisgruppe \pageref{Grenzkreisgruppe} 

\bigskip
\textbf{H} \\
\\
\textsc{Hadamard}scher Satz \"uber das Geschlecht ganzer
Funktionen \pageref{Hadamard} \\
\textsc{Hilbert} \pageref{Hilbert1}, \pageref{Hilbert2} \\
\textsc{Hill} \pageref{Hill} 

\bigskip
\textbf{I} \\
\\
Integralgleichung 1.~Art \pageref{Integral1Art} \\
--- 2. Art \pageref{Integral2Art} 

\bigskip
\textbf{K} \\
\\
\textsc{Kaufmann} \pageref{Kaufmann} \\
\textsc{Kellogg} \pageref{Kellog} \\
Kerne, iterierte \pageref{itKerne1}, \pageref{itKerne2}\\
--- unendliche \pageref{unendlKern1}, \pageref{unendlKern2} \\
\textsc{Koch}, Helge von --- \pageref{Koch} \\
Konduktionsstrom \pageref{Konduk} \\
Korrespondenz auf algebr.~Kurven \pageref{Korrespondenz} \\
Kontinuumproblem \pageref{Kontinuum} 

\bigskip
\textbf{L} \\
\\
\textsc{Legendre}sche Polynome \pageref{Legendre} \\
\textsc{Leverrier} \pageref{Leverrier} \\
Lichtgeschwindigkeit \pageref{Lichtgeschwindig} \\
lineare Gleichungen mit unendlichvielen Unbekannten \pageref{linGl} \\
\textsc{Lorentz}, H.\ A. \pageref{Lorentz} \\


\bigskip
\textbf{M} \\
\\
magnetische Kraft \pageref{Magnet} \\
\textsc{Maxwell} \pageref{Maxwell} \\
Mercur \pageref{Mercur1}, \pageref{Mercur2} \\
\textsc{Michelson} \pageref{Michelson} 

\bigskip
\textbf{N} \\
\\
nichteuklidische Geometrie \pageref{nichteuklid} ff. 

\bigskip
\textbf{P} \\
\\
Periodentabelle \pageref{Periodentabelle} \\
\textsc{Picard} \pageref{Picard} \\
Potential, hydrostatisches \pageref{hydroPot} \\
--- retardiertes \pageref{retardPot} \\
--- skalares \pageref{skalaresPot} \\
pr\"adikative Definitionen \pageref{praedikativ} 

\bigskip
\textbf{R} \\
\\
Reduktionstheorie \textsc{Abel}scher Integrale 
\pageref{Reduktionstheorie} \\
\textsc{Richard} \pageref{Richard} ff. \\
\textsc{R\"ontgen}strahlen \pageref{Roentgen} \\
\textsc{Russell} \pageref{Russell} 

\bigskip
\textbf{S} \\
\\
\textsc{Schoenflies} \pageref{Schoenflies} \\
Schwingung, ged\"ampfte synchrone \pageref{synchron} 

\bigskip
\textbf{T} \\
\\
temps apparent  \pageref{tempsapparent} 

\bigskip
\textbf{W} \\
\\
Wurzelexistenzbeweis \pageref{Wurzeln} 

\bigskip
\textbf{Z} \\
\\
Zahlen, charakteristische \pageref{charakteristische} \\
--- invariante \pageref{invariante} \\
--- transfinite \pageref{transfinite} \\
\textsc{Zermelo} \pageref{Zermelo} \\
\vfill

\newpage
%\chapter{PROJECT GUTENBERG ``SMALL PRINT''}
\small 
\pagenumbering{gobble}
\begin{verbatim}

*** END OF THIS PROJECT GUTENBERG EBOOK ***

***** This file should be named 15267-t.tex or 15267-t.zip *****
This and all associated files of various formats will be found in:
        http://www.gutenberg.org/1/5/2/6/15267/


Produced by Joshua Hutchinson, K.F. Creiner and the Online Distributed
Proofreading Team. This file was produced from images generously made
available by Cornell University.


Updated editions will replace the previous one--the old editions
will be renamed.

Creating the works from public domain print editions means that no
one owns a United States copyright in these works, so the Foundation
(and you!) can copy and distribute it in the United States without
permission and without paying copyright royalties.  Special rules,
set forth in the General Terms of Use part of this license, apply to
copying and distributing Project Gutenberg-tm electronic works to
protect the PROJECT GUTENBERG-tm concept and trademark.  Project
Gutenberg is a registered trademark, and may not be used if you
charge for the eBooks, unless you receive specific permission.  If
you do not charge anything for copies of this eBook, complying with
the rules is very easy.  You may use this eBook for nearly any
purpose such as creation of derivative works, reports, performances
and research.  They may be modified and printed and given away--you
may do practically ANYTHING with public domain eBooks.
Redistribution is subject to the trademark license, especially
commercial redistribution.



*** START: FULL LICENSE ***

THE FULL PROJECT GUTENBERG LICENSE PLEASE READ THIS BEFORE YOU
DISTRIBUTE OR USE THIS WORK

To protect the Project Gutenberg-tm mission of promoting the free
distribution of electronic works, by using or distributing this work
(or any other work associated in any way with the phrase "Project
Gutenberg"), you agree to comply with all the terms of the Full
Project Gutenberg-tm License (available with this file or online at
http://gutenberg.net/license).


Section 1.  General Terms of Use and Redistributing Project
Gutenberg-tm electronic works

1.A.  By reading or using any part of this Project Gutenberg-tm
electronic work, you indicate that you have read, understand, agree
to and accept all the terms of this license and intellectual
property (trademark/copyright) agreement.  If you do not agree to
abide by all the terms of this agreement, you must cease using and
return or destroy all copies of Project Gutenberg-tm electronic
works in your possession. If you paid a fee for obtaining a copy of
or access to a Project Gutenberg-tm electronic work and you do not
agree to be bound by the terms of this agreement, you may obtain a
refund from the person or entity to whom you paid the fee as set
forth in paragraph 1.E.8.

1.B.  "Project Gutenberg" is a registered trademark.  It may only be
used on or associated in any way with an electronic work by people
who agree to be bound by the terms of this agreement.  There are a
few things that you can do with most Project Gutenberg-tm electronic
works even without complying with the full terms of this agreement.
See paragraph 1.C below.  There are a lot of things you can do with
Project Gutenberg-tm electronic works if you follow the terms of
this agreement and help preserve free future access to Project
Gutenberg-tm electronic works.  See paragraph 1.E below.

1.C.  The Project Gutenberg Literary Archive Foundation ("the
Foundation" or PGLAF), owns a compilation copyright in the
collection of Project Gutenberg-tm electronic works.  Nearly all the
individual works in the collection are in the public domain in the
United States.  If an individual work is in the public domain in the
United States and you are located in the United States, we do not
claim a right to prevent you from copying, distributing, performing,
displaying or creating derivative works based on the work as long as
all references to Project Gutenberg are removed.  Of course, we hope
that you will support the Project Gutenberg-tm mission of promoting
free access to electronic works by freely sharing Project
Gutenberg-tm works in compliance with the terms of this agreement
for keeping the Project Gutenberg-tm name associated with the work.
You can easily comply with the terms of this agreement by keeping
this work in the same format with its attached full Project
Gutenberg-tm License when you share it without charge with others.

1.D.  The copyright laws of the place where you are located also
govern what you can do with this work.  Copyright laws in most
countries are in a constant state of change.  If you are outside the
United States, check the laws of your country in addition to the
terms of this agreement before downloading, copying, displaying,
performing, distributing or creating derivative works based on this
work or any other Project Gutenberg-tm work.  The Foundation makes
no representations concerning the copyright status of any work in
any country outside the United States.

1.E.  Unless you have removed all references to Project Gutenberg:

1.E.1.  The following sentence, with active links to, or other
immediate access to, the full Project Gutenberg-tm License must
appear prominently whenever any copy of a Project Gutenberg-tm work
(any work on which the phrase "Project Gutenberg" appears, or with
which the phrase "Project Gutenberg" is associated) is accessed,
displayed, performed, viewed, copied or distributed:

This eBook is for the use of anyone anywhere at no cost and with
almost no restrictions whatsoever.  You may copy it, give it away or
re-use it under the terms of the Project Gutenberg License included
with this eBook or online at www.gutenberg.net

1.E.2.  If an individual Project Gutenberg-tm electronic work is
derived from the public domain (does not contain a notice indicating
that it is posted with permission of the copyright holder), the work
can be copied and distributed to anyone in the United States without
paying any fees or charges.  If you are redistributing or providing
access to a work with the phrase "Project Gutenberg" associated with
or appearing on the work, you must comply either with the
requirements of paragraphs 1.E.1 through 1.E.7 or obtain permission
for the use of the work and the Project Gutenberg-tm trademark as
set forth in paragraphs 1.E.8 or 1.E.9.

1.E.3.  If an individual Project Gutenberg-tm electronic work is
posted with the permission of the copyright holder, your use and
distribution must comply with both paragraphs 1.E.1 through 1.E.7
and any additional terms imposed by the copyright holder.
Additional terms will be linked to the Project Gutenberg-tm License
for all works posted with the permission of the copyright holder
found at the beginning of this work.

1.E.4.  Do not unlink or detach or remove the full Project
Gutenberg-tm License terms from this work, or any files containing a
part of this work or any other work associated with Project
Gutenberg-tm.

1.E.5.  Do not copy, display, perform, distribute or redistribute
this electronic work, or any part of this electronic work, without
prominently displaying the sentence set forth in paragraph 1.E.1
with active links or immediate access to the full terms of the
Project Gutenberg-tm License.

1.E.6.  You may convert to and distribute this work in any binary,
compressed, marked up, nonproprietary or proprietary form, including
any word processing or hypertext form.  However, if you provide
access to or distribute copies of a Project Gutenberg-tm work in a
format other than "Plain Vanilla ASCII" or other format used in the
official version posted on the official Project Gutenberg-tm web
site (www.gutenberg.net), you must, at no additional cost, fee or
expense to the user, provide a copy, a means of exporting a copy, or
a means of obtaining a copy upon request, of the work in its
original "Plain Vanilla ASCII" or other form.  Any alternate format
must include the full Project Gutenberg-tm License as specified in
paragraph 1.E.1.

1.E.7.  Do not charge a fee for access to, viewing, displaying,
performing, copying or distributing any Project Gutenberg-tm works
unless you comply with paragraph 1.E.8 or 1.E.9.

1.E.8.  You may charge a reasonable fee for copies of or providing
access to or distributing Project Gutenberg-tm electronic works
provided that

- You pay a royalty fee of 20% of the gross profits you derive from
     the use of Project Gutenberg-tm works calculated using the method
     you already use to calculate your applicable taxes.  The fee is
     owed to the owner of the Project Gutenberg-tm trademark, but he
     has agreed to donate royalties under this paragraph to the
     Project Gutenberg Literary Archive Foundation.  Royalty payments
     must be paid within 60 days following each date on which you
     prepare (or are legally required to prepare) your periodic tax
     returns.  Royalty payments should be clearly marked as such and
     sent to the Project Gutenberg Literary Archive Foundation at the
     address specified in Section 4, "Information about donations to
     the Project Gutenberg Literary Archive Foundation."

- You provide a full refund of any money paid by a user who notifies
     you in writing (or by e-mail) within 30 days of receipt that s/he
     does not agree to the terms of the full Project Gutenberg-tm
     License.  You must require such a user to return or
     destroy all copies of the works possessed in a physical medium
     and discontinue all use of and all access to other copies of
     Project Gutenberg-tm works.

- You provide, in accordance with paragraph 1.F.3, a full refund of
     any money paid for a work or a replacement copy, if a defect in
     the electronic work is discovered and reported to you within 90
     days of receipt of the work.

- You comply with all other terms of this agreement for free
     distribution of Project Gutenberg-tm works.

1.E.9.  If you wish to charge a fee or distribute a Project
Gutenberg-tm electronic work or group of works on different terms
than are set forth in this agreement, you must obtain permission in
writing from both the Project Gutenberg Literary Archive Foundation
and Michael Hart, the owner of the Project Gutenberg-tm trademark.
Contact the Foundation as set forth in Section 3 below.

1.F.

1.F.1.  Project Gutenberg volunteers and employees expend
considerable effort to identify, do copyright research on,
transcribe and proofread public domain works in creating the Project
Gutenberg-tm collection.  Despite these efforts, Project
Gutenberg-tm electronic works, and the medium on which they may be
stored, may contain "Defects," such as, but not limited to,
incomplete, inaccurate or corrupt data, transcription errors, a
copyright or other intellectual property infringement, a defective
or damaged disk or other medium, a computer virus, or computer codes
that damage or cannot be read by your equipment.

1.F.2.  LIMITED WARRANTY, DISCLAIMER OF DAMAGES - Except for the
"Right of Replacement or Refund" described in paragraph 1.F.3, the
Project Gutenberg Literary Archive Foundation, the owner of the
Project Gutenberg-tm trademark, and any other party distributing a
Project Gutenberg-tm electronic work under this agreement, disclaim
all liability to you for damages, costs and expenses, including
legal fees.  YOU AGREE THAT YOU HAVE NO REMEDIES FOR NEGLIGENCE,
STRICT LIABILITY, BREACH OF WARRANTY OR BREACH OF CONTRACT EXCEPT
THOSE PROVIDED IN PARAGRAPH F3.  YOU AGREE THAT THE FOUNDATION, THE
TRADEMARK OWNER, AND ANY DISTRIBUTOR UNDER THIS AGREEMENT WILL NOT
BE LIABLE TO YOU FOR ACTUAL, DIRECT, INDIRECT, CONSEQUENTIAL,
PUNITIVE OR INCIDENTAL DAMAGES EVEN IF YOU GIVE NOTICE OF THE
POSSIBILITY OF SUCH DAMAGE.

1.F.3.  LIMITED RIGHT OF REPLACEMENT OR REFUND - If you discover a
defect in this electronic work within 90 days of receiving it, you
can receive a refund of the money (if any) you paid for it by
sending a written explanation to the person you received the work
from.  If you received the work on a physical medium, you must
return the medium with your written explanation.  The person or
entity that provided you with the defective work may elect to
provide a replacement copy in lieu of a refund.  If you received the
work electronically, the person or entity providing it to you may
choose to give you a second opportunity to receive the work
electronically in lieu of a refund.  If the second copy is also
defective, you may demand a refund in writing without further
opportunities to fix the problem.

1.F.4.  Except for the limited right of replacement or refund set
forth in paragraph 1.F.3, this work is provided to you 'AS-IS', WITH
NO OTHER WARRANTIES OF ANY KIND, EXPRESS OR IMPLIED, INCLUDING BUT
NOT LIMITED TO WARRANTIES OF MERCHANTIBILITY OR FITNESS FOR ANY
PURPOSE.

1.F.5.  Some states do not allow disclaimers of certain implied
warranties or the exclusion or limitation of certain types of
damages. If any disclaimer or limitation set forth in this agreement
violates the law of the state applicable to this agreement, the
agreement shall be interpreted to make the maximum disclaimer or
limitation permitted by the applicable state law.  The invalidity or
unenforceability of any provision of this agreement shall not void
the remaining provisions.

1.F.6.  INDEMNITY - You agree to indemnify and hold the Foundation,
the trademark owner, any agent or employee of the Foundation, anyone
providing copies of Project Gutenberg-tm electronic works in
accordance with this agreement, and any volunteers associated with
the production, promotion and distribution of Project Gutenberg-tm
electronic works, harmless from all liability, costs and expenses,
including legal fees, that arise directly or indirectly from any of
the following which you do or cause to occur: (a) distribution of
this or any Project Gutenberg-tm work, (b) alteration, modification,
or additions or deletions to any Project Gutenberg-tm work, and (c)
any Defect you cause.


Section  2.  Information about the Mission of Project Gutenberg-tm

Project Gutenberg-tm is synonymous with the free distribution of
electronic works in formats readable by the widest variety of
computers including obsolete, old, middle-aged and new computers.
It exists because of the efforts of hundreds of volunteers and
donations from people in all walks of life.

Volunteers and financial support to provide volunteers with the
assistance they need, is critical to reaching Project Gutenberg-tm's
goals and ensuring that the Project Gutenberg-tm collection will
remain freely available for generations to come.  In 2001, the
Project Gutenberg Literary Archive Foundation was created to provide
a secure and permanent future for Project Gutenberg-tm and future
generations. To learn more about the Project Gutenberg Literary
Archive Foundation and how your efforts and donations can help, see
Sections 3 and 4 and the Foundation web page at
http://www.pglaf.org.


Section 3.  Information about the Project Gutenberg Literary Archive
Foundation

The Project Gutenberg Literary Archive Foundation is a non profit
501(c)(3) educational corporation organized under the laws of the
state of Mississippi and granted tax exempt status by the Internal
Revenue Service.  The Foundation's EIN or federal tax identification
number is 64-6221541.  Its 501(c)(3) letter is posted at
http://pglaf.org/fundraising.  Contributions to the Project
Gutenberg Literary Archive Foundation are tax deductible to the full
extent permitted by U.S. federal laws and your state's laws.

The Foundation's principal office is located at 4557 Melan Dr. S.
Fairbanks, AK, 99712., but its volunteers and employees are
scattered throughout numerous locations.  Its business office is
located at 809 North 1500 West, Salt Lake City, UT 84116, (801)
596-1887, email business@pglaf.org.  Email contact links and up to
date contact information can be found at the Foundation's web site
and official page at http://pglaf.org

For additional contact information:
     Dr. Gregory B. Newby
     Chief Executive and Director
     gbnewby@pglaf.org

Section 4.  Information about Donations to the Project Gutenberg
Literary Archive Foundation

Project Gutenberg-tm depends upon and cannot survive without wide
spread public support and donations to carry out its mission of
increasing the number of public domain and licensed works that can
be freely distributed in machine readable form accessible by the
widest array of equipment including outdated equipment.  Many small
donations ($1 to $5,000) are particularly important to maintaining
tax exempt status with the IRS.

The Foundation is committed to complying with the laws regulating
charities and charitable donations in all 50 states of the United
States.  Compliance requirements are not uniform and it takes a
considerable effort, much paperwork and many fees to meet and keep
up with these requirements.  We do not solicit donations in
locations where we have not received written confirmation of
compliance.  To SEND DONATIONS or determine the status of compliance
for any particular state visit http://pglaf.org

While we cannot and do not solicit contributions from states where
we have not met the solicitation requirements, we know of no
prohibition against accepting unsolicited donations from donors in
such states who approach us with offers to donate.

International donations are gratefully accepted, but we cannot make
any statements concerning tax treatment of donations received from
outside the United States.  U.S. laws alone swamp our small staff.

Please check the Project Gutenberg Web pages for current donation
methods and addresses.  Donations are accepted in a number of other
ways including including checks, online payments and credit card
donations.  To donate, please visit: http://pglaf.org/donate


Section 5.  General Information About Project Gutenberg-tm
electronic works.

Professor Michael S. Hart is the originator of the Project
Gutenberg-tm concept of a library of electronic works that could be
freely shared with anyone.  For thirty years, he produced and
distributed Project Gutenberg-tm eBooks with only a loose network of
volunteer support.

Project Gutenberg-tm eBooks are often created from several printed
editions, all of which are confirmed as Public Domain in the U.S.
unless a copyright notice is included.  Thus, we do not necessarily
keep eBooks in compliance with any particular paper edition.

Most people start at our Web site which has the main PG search
facility:

     http://www.gutenberg.net

This Web site includes information about Project Gutenberg-tm,
including how to make donations to the Project Gutenberg Literary
Archive Foundation, how to help produce our new eBooks, and how to
subscribe to our email newsletter to hear about new eBooks.
\end{verbatim}

\end{document}
